\chapter*{Conclusions}
\chaptermark{Conclusions}
\addcontentsline{toc}{chapter}{Conclusions}

Tout au long de ce travail de thèse, nous avons suivi un thème central : développer des méthodes capables de prendre en compte et de tirer parti des similitudes entre différentes maladies transmises par les moustiques du genre {\em Aedes}.
Cette approche a en partie découlé de la disponibilité de données sur dix-huit épidémies de chikungunya et de Zika dans six îles ou petits archipels de Polynésie française et trois îles des Antilles françaises entre 2013 et 2017.
Ces territoires conjuguent en effet plusieurs caractéristiques importantes : les épidémies de chikungunya et de Zika s'y sont produites successivement dans un intervalle de temps d'un ou deux ans, les données d'incidence y ont été collectées par un système de surveillance bien organisé et stable au cours du temps, et enfin il s'agit d'îles relativement petites, ce qui permet certaines hypothèses facilitant grandement l'analyse.
Nous avons dans un premier temps conduit un travail d'épidémiologie comparée de ces épidémies.
Pour cela, nous avons développé un modèle sophistiqué, capable de modéliser conjointement l'incidence hebdomadaire observée durant l'ensemble des épidémies.
Ce modèle avait une composante hiérarchique, ce qui est une des approches les plus adaptées pour séparer des caractéristiques locales de caractéristiques générales. 
L'hétérogénéité des dynamiques épidémiques entre les territoires persistait après la prise en compte de l'effet de la maladie en elle-même sur la transmission et le signalement des cas, et de l'effet des conditions météorologiques locales.
Cela souligne l'importance des caractéristiques locales sur les processus de transmission et de signalement de ces maladies, bien que l'échelle utilisée ici ne permette pas d'aller plus loin dans l'identification des facteurs locaux les plus importants.
Ayant quantifié les similarités et les divergences entre les dynamiques épidémiques du chikungunya et du Zika, nous avons voulu savoir s'il était possible d'utiliser ces connaissances pour apporter des solutions au problème de l'évaluation épidémiologique empirique en temps réel en situation d'émergence.
Nous nous sommes donc mis à la place d'épidémiologistes cherchant à évaluer précocement le potentiel de l'épidémie de Zika des Antilles françaises, dans les premiers mois de l'année 2016.
A ce moment, les données concernant les épidémies de chikungunya dans les mêmes îles (2013-2014), ainsi que les données concernant les épidémies successives de Zika et de chikungunya de Polynésie française (2013-2015) étaient disponibles et auraient pu être utilisées.
Utilisant des variations du modèle hiérarchique développé pour le travail précédent, nous avons pu montrer qu'il était possible d'intégrer ces informations historiques dans les modèles de prédiction, et d'améliorer ainsi la fiabilité de ces évaluations.
Ces résultats montrent l'intérêt des approches conjointes, considérant ensemble plusieurs épidémies de maladies différentes mais proches dans plusieurs territoires.
Cela permet en effet de pouvoir directement comparer les effets de certains facteurs sur les dynamiques épidémiques, et d'obtenir une meilleure compréhension des processus impliqués.
De plus, les modèles hiérarchiques adaptés à ces circonstances permettent de mettre en commun diverses sources d'information, et d'obtenir ainsi des estimations des paramètres épidémiques plus fiables et plus généralisables, qui peuvent ensuite être utilisé pour soutenir les prédictions réalisées lors d'éventuelles émergences épidémiques ultérieures.
De façon générale, il semble important de diriger une partie des efforts de recherche sur les maladies émergentes vers la compilation systématique de données épidémiques, et vers des travaux d'épidémiologie comparée entre maladies similaires.

