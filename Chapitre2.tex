\chapter{Aspects méthodologiques}
\chaptermark{}

L'existence de fortes concentrations de moustiques du genre {\em Aedes} à proximité de centres de population humaine dans les zones tropicales et tempérées exerce une pression de sélection favorisant l'émergence de maladies humaines transmises par ces vecteurs.
Nous avons passé en revue au chapitre 1 les origines de cette situation et les conséquences déjà observées.
Malgré de récents développements dans les stratégies de lutte antivectorielle, il est probable que de nouvelles émergences vont se produire, avec des conséquences imprévisibles sur la santé des populations.
Au delà de l'étude individuelle des épidémies passées, qui reste un sujet important, l'observation plus générale des dynamiques épidémiques caractéristiques des maladies transmises par les moustiques du genre {\em Aedes} pourrait permettre de mieux anticiper et contrôler les émergences futures.
Avec cet objectif, nous proposons une approche innovante consistant à comparer directement plusieurs épidémies partageant le même mode de transmission et touchant les mêmes populations.
Nous nous sommes concentrés sur l'analyse des épidémies successives de Chikungunya et de Zika ayant eu lieu en Polynésie française et aux Antilles françaises entre 2013 et 2017 au moyen d'un modèle unique hiérarchique, permettant de distinguer les influences respectives de plusieurs facteurs sur les niveaux de transmission.
Le développement de ce modèle nous a conduit à emprunter des concepts venant de plusieurs champs de l'épidémiologie et des biostatistiques, que nous passerons en revue dans ce chapitre.

\section{Les modèles de maladies vectorielles}

Les modèles de maladies vectorielles trouvent leur origine dans les travaux de Ronald Ross sur le paludisme, qui développa une approche mathématique incluant le cycle complet de transmission d'un pathogène entre populations d'hôtes et populations de vecteurs, ainsi que les relations entre ces entités. 
Cette approche, étendue et formalisée par George Macdonald, est toujours très influente aujourd'hui, et constitue l'aboutissement d'une théorie plus générale des dynamiques épidémiques et du contrôle des maladies transmises par les moustiques qui se développe à partir de la fin du XIX\textsuperscript{ème} siècle \cite{smith2012ross}.
Nous nous proposons dans une première partie de retracer brièvement les avancées qui ont conduit aux modèles utilisés de nos jours.

\subsection{Ross, Macdonald et le développement des modèles de transmission du paludisme}

Au co\oe ur de la révolution microbiologique, Patrick Manson isole en 1877 le pathogène responsable de la filariose lymphatique dans des moustiques ayant piqué des malades en Chine, mettant en lumière le rôle possible d'invertébrés en tant que vecteurs de maladies humaines. 
A la suite d'Alphonse Laveran, qui identifie le parasite du paludisme, Ronald Ross démontre que cette maladie est transmise par les anophèles femelles en 1889. 
Ross est le premier à faire le lien entre l'épidémiologie du paludisme dans les populations humaines et les relations complexes entre parasites, hôtes et vecteurs, et entreprend de synthétiser ces relations en utilisant des outils mathématiques.
Dès 1908, Ross conçoit un premier modèle de transmission du paludisme \cite{ross1908report}.
Ce modèle initial est réexprimé par Alfred James Lotka sous la forme d'une suite récurrente reliant le nombre d'humains infectés au temps $t+1$, noté $I_{t+1}$, au nombre d'infectés au temps $t$ selon
\begin{equation}
I_{t+1} = \hat{V}\frac{I_t}{N}(N-I_t)-rI_t,
\end{equation}
où $N$ est le nombre total d'humains, $r$ le taux de guérison et $\hat{V}$ est une mesure similaire à la capacité vectorielle, qui résume à la fois le nombre et l'activité des moustiques. 
Ce modèle met en évidence la relation non-linéaire entre le nombre de vecteurs et l'intensité de la transmission. 
Il suggère qu'il n'est pas nécessaire d'éliminer tous les moustiques pour contrôler la maladie, mais qu'il existe une population limite en dessous de laquelle la transmission soutenue du paludisme n'est plus possible, ce qui a des conséquences importantes pour les stratégies de lutte antivectorielles qui commencent à se développer à cette époque.
Les travaux de Ross furent vite reconnus, et ont largement contribué au développement de l'épidémiologique quantitative, notamment influençant directement William Kermack et Anderson Mackendrick qui publient en 1927 leur théorie mathématique des épidémies, qui mènera au modèle SIR \cite{kermack1927contributions}.

Ces recherches furent poursuivies dans les années 1950 par George Macdonald, dans le contexte du lancement du programme d'éradication globale du paludisme par l'OMS.
S'appuyant sur les travaux de Ross, Macdonald aboutit à la formulation d'un modèle reflétant directement le cycle biologique du parasite \cite{macdonald1952analysis,koella1991use}.
Considérons d'abord la transmission du parasite des vecteurs aux hôtes : si chaque anophèle femelle pique un humain $a$ fois par jour, et qu'il existe une densité de $m$ anophèles femelles par humain, chaque humain est piqué $ma$ fois par jour.
Si le parasite est présent dans les glandes salivaires d'une fraction $z$ des vecteurs, et que chaque piqûre a une probabilité $b$ de transmettre le parasite, cela réduit le nombre de piqûres infectieuses par hôte et par jour à $mabz$.
Si on fait l'hypothèse qu'une infection ne peut se produire que chez un hôte non-encore infecté, et si la proportion d'hôtes infectés est $w$, alors $w$ augmente chaque jour de $mabz(1-w)$.
Si une fois infectés, les humains guérissent à un taux $r$, c'est à dire que la durée moyenne de l'infection est de $1/r$ jours, la variation de la proportion d'infectés parmi les hôtes dans le temps peut être exprimée par l'équation différentielle :
\begin{equation}
\label{eq:eqRM1}
\frac{dw}{dt} = mabz(1-w) - rw.
\end{equation}
Considérons maintenant la transmission du parasite des hôtes vers les vecteurs.
La population des vecteurs peut être divisée en trois catégories : $x$ la proportion de vecteurs susceptibles, $z$ la proportion de vecteurs dont les glandes salivaires sont infectées par le parasite, et $y$ la proportion de vecteurs infectés mais latents, c'est à dire chez qui le parasite n'a pas encore atteint les glandes salivaires ($x+y+z=1$).
Suivant un raisonnement similaire, les moustiques susceptibles piquent chacun $a$ hôtes par jour, une proportion $w$ de ces hôtes sont porteurs du parasite, et une proportion $c$ des piqûres potentiellement infectieuses causent effectivement une infection, ce qui fait que la proportion de vecteurs latents $y$ augmente chaque jour de $acw(1-y-z)$.
Ces vecteurs nouvellement infectés deviennent à leur tour infectieux en une durée $v$, le temps que le parasite complète son cycle et atteigne les glandes salivaires (aussi nommée durée d'incubation extrinsèque), s'ils survivent jusque là.
Si $g$ est la mortalité des moustiques, alors une proportion $e^{-vg}$ des moustiques latents survivent assez longtemps pour devenir infectieux.
Les variations de $y$ et de $z$ dans le temps peuvent être résumées par les équations suivantes :
\begin{align}
\frac{dy}{dt} &= acw(1-y-z) - ac{w'}(1-{y'}-{z'})e^{-vg} - gy \\
\frac{dz}{dt} &= ac{w'}(1-{y'}-{z'})e^{-vg} - gz, 
\end{align}
où ${w'}$, ${y'}$ et ${z'}$ correspondent respectivement à $w$, $y$ et $z$ au temps $t-v$.
Ce système de trois équations différentielles constitue la base des modèles de type Ross-Macdonald, dont il n'existe pas une formulation fixe, mais plutôt un ensemble de modèles suivant un certain nombre d'hypothèses simplificatrices \cite{smith2012ross} : 
\begin{itemize}
\item on considère un seul type de pathogène, un seul type d'hôte et un seul type de vecteur, dont les contacts sont pris en compte explicitement;
\item on considère une zone géographique donnée, sans émigration ni immigration;
\item la valeur des paramètres est constante au cours du temps, les durées ont une distribution exponentielle;
\item le cycle aquatique du vecteur n'est pas pris en compte explicitement;
\item la distribution des piqûres parmi les hôtes est homogène;
\item les populations d'hôtes et de vecteurs sont homogènes;
\item l'immunité acquise chez l'hôte n'est pas prise en compte;
\item la coinfection ou la superinfection des hôtes n'est pas prise en compte.
\end{itemize}

En plus de compléter les travaux de Ross, Macdonald apporta des avancées  conceptuelles majeures, faisant le lien entre plusieurs domaines scientifiques.
Il emprunta notamment à Lotka la notion démographique de {\em taux de reproduction de base}, qu'il nomma d'abord $Z_0$, et qui deviendra $\mathcal{R}_0$, dont la formule peut être dérivée analytiquement du modèle présenté ci-dessus :
\begin{equation}
\mathcal{R}_0 = \frac{ma^2bc}{gr}e^{-vg}.
\end{equation}
$\mathcal{R}_0$ est une mesure du nombre attendu de cas humains secondaires infectés par un seul cas index dans une population entièrement susceptible, et représente donc une mesure de l'intensité de la transmission.
Sa formule peut être interprétée intuitivement : la transmission du paludisme est favorisée par une densité élevée de moustiques ($m$ élevé) qui piquent fréquemment ($a$ élevé) et une grande susceptibilité à l'infection des vecteurs ($c$ élevé) et des hôtes ($b$ élevé).
Au contraire, la transmission est affaiblie par une guérison plus rapide des hôtes ($r$ élevé) et une plus haute mortalité des vecteurs ($g$ élevé).
Puisque deux piqûres sont nécessaires pour compléter le cycle de transmission du parasite, le terme $a$ intervient élevé au carré, et constitue donc une cible privilégiée d'intervention : ainsi, diviser la densité de moustiques $m$ par deux, par exemple par la dispersion de larvicides, réduit théoriquement $\mathcal{R}_0$ d'un facteur deux, mais diviser le nombre de piqûres par deux, par exemple au moyen de moustiquaires, réduit $\mathcal{R}_0$ d'un facteur quatre.

L'analyse des points d'équilibre du système montre la relation entre $\mathcal{R}_0$ et la prévalence à l'équilibre chez les hôtes :
\begin{equation}
\label{eq:equi1}
\hat{w} = \frac{\mathcal{R}_0-1}{\mathcal{R}_0-\frac{a}{g}};
\end{equation}
et chez les vecteurs :
\begin{equation}
\label{eq:equi2}
\hat{z} = \frac{\mathcal{R}_0-1}{\mathcal{R}_0}\frac{\frac{a}{g}}{1+\frac{a}{g}}e^{-vg}.
\end{equation}
Ces valeurs sont positives seulement si $\mathcal{R}_0>1$, ce qui correspond à l'affirmation de Ross selon laquelle il existe une limite en dessous de laquelle la transmission soutenue du paludisme n'est plus possible (Fig. \ref{fig:equilibrium}).

\begin{figure}[t]
	\centering
	\includegraphics[width=12cm]{Figures/ross_macdonald_equilibrium.PNG}
	\caption{Prévalences à l'équilibre chez les humains et les moustiques en fonction de $\mathcal{R}_0$ selon le modèle de Ross-Macdonald présenté suivant les équations \ref{eq:equi1} et \ref{eq:equi2} (source : Koella, 1991).}
	\label{fig:equilibrium}
\end{figure}


Macdonald proposa aussi des méthodes de mesure entomologique de la transmission qui mèneront au concept de {\em capacité vectorielle}, défini comme le nombre attendu de piqûres potentiellement infectieuses découlant de l'existence d'un seul cas humain infectieux en contact avec une population de vecteurs 
\begin{equation}
V = \frac{ma^2}{g}e^{-gv}.
\end{equation}
Pendant de nombreuses années, le risque de paludisme dans une région donnée sera évalué suivant cette approche, par des mesures entomologiques sur le terrain des différents paramètres permettant de calculer la capacité vectorielle.

Des adaptations ont été apportées au modèle par la suite, suivant l'évolution des besoins et des connaissances biologiques et entomologiques.
Pour autant, les hypothèses et la structure des modèles de type Ross-Macdonald restent d'actualité. 
Une revue systématiques a ainsi rapporté que plus de la moitié des modèles de maladies vectorielles publiés entre 1970 et 2010 ne déviaient pratiquement pas de cette approche \cite{reiner_systematic_2013}.
C'est surtout dans l'utilisation qui est faite des méthodes de modélisation qu'une évolution a été visible, en lien avec le développement des systèmes de surveillance épidémiologique.
Initialement, les modèles étaient surtout utilisés comme des outils théoriques, avec pour objectif de mieux comprendre la transmission et de cibler les mesures de prévention et de contrôle, ou bien pour une estimation de type qualitatif d'un risque d'épidémie en se basant sur des mesures entomologiques.
L'abondance et la relative fiabilité des données d'incidence ou de séroprévalence dans les populations humaines, contrastant avec la difficulté des mesures entomologiques, ont entraîné une modification des pratiques, avec pour objectifs premiers l'estimation directe des dynamiques épidémiques, en particulier par la mesure du taux de reproduction de base $\mathcal{R}_0$, la quantification des facteurs influençant ces dynamiques, et dans certains cas la prédiction ou la simulation d'épidémies en population.

\begin{figure}[t]
	\centering
	\includegraphics[width=14cm]{Figures/reiner_diseases.PNG}
	\caption{Evolution du nombre de modèles de maladies vectorielles publiés entre 1970 et 2010 selon la maladie (source : Reiner et al, 2013)}
	\label{fig:reiner_diseases}
\end{figure}


\subsection{Extension aux maladies transmises par les moustiques du genre {\em Aedes}}

Les théories attachées aux modèles de Ross-Macdonald furent progressivement appliquées à d'autres maladies que le paludisme (Fig. \ref{fig:reiner_diseases}) \cite{reiner_systematic_2013}.
Sur le sujet des maladies transmises par les moustiques du genre {\em Aedes}, une première tentative d'adaptation du modèle de Ross-Macdonald à la transmission d'un seul sérotype de dengue fut proposée par Bailey \cite{bailey1975mathematical,andraud2012dynamic}.
La formulation du modèle reste proche de celle proposée par Macdonald, la principale adaptation consistant en la prise en compte d'une immunité acquise à long terme chez l'hôte après infection, qui n'existe pas pour le paludisme. 
On observe aussi des différences liées à l'influence des modèles de type SIR développés pour les maladies transmises directement d'hôte à hôte.
Ainsi on préfère considérer le nombre d'hôtes ou de vecteurs dans chaque compartiment plutôt que la proportion : la population d'hôtes de taille $N$ est divisée en trois compartiments (nombre de susceptibles $S$, d'infectieux $I$ et d'immunisés ou résistants $R$) et la population de vecteurs de taille $M$ est divisée en deux compartiments (nombre de susceptibles $X$ et d'infectieux $Z$).
De ce fait, plutôt que le taux d'augmentation de la proportion d'hôtes infectés $mabz(1-w)$ apparaissant dans l'équation \ref{eq:eqRM1}, on considère le taux d'augmentation du nombre d'hôtes infectés, qui peut être reformulé selon :
\begin{equation}
mabz(1-w)N = \frac{M}{N}ab\frac{Z}{M}(1-\frac{I}{N})N = ab\frac{SZ}{N}.
\end{equation}
On peut remarquer que cette formule n'est pas sans rappeler le terme $\beta\frac{SI}{N}$ intervenant dans le modèle SIR.
Le modèle est décrit par le système suivant :
\begin{align}
\frac{dS}{dt} &= fN - ab\frac{SZ}{N} - fS \\ \nonumber
\frac{dI}{dt} &= ab\frac{SZ}{N} - rI - fI \\ \nonumber
\frac{dR}{dt} &= rI - fR \\  \nonumber
\frac{dX}{dt} &= A - ac\frac{XI}{N} - gX \\ \nonumber
\frac{dZ}{dt} &= ac\frac{XI}{N} - gZ
\end{align}
où $f$ désigne à la fois le taux de mortalité et de natalité des hôtes, et $A$ le taux (constant) de recrutement de nouveaux moustiques adultes.
Les autres symboles correspondent à ceux utilisés dans le premier modèle : $a$ est le nombre de piqûres par moustique et par unité de temps, $b$ la probabilité que la piqûre de moustique infectieux résulte en une infection chez l'homme, $c$  la probabilité que la piqûre d'un humain infectieux résulte en une infection chez le moustique, $r$ est le taux de guérison chez l'humain et $g$ le taux de mortalité chez le moustique.

Ce modèle fut utilisé pour étudier l'efficacité des pulvérisations d'insecticides en \guillemotleft ultra-bas volume\guillemotright à l'aide de simulations.
Il a été à la base d'un grand nombre de modèles de maladies transmises par les moustiques du genre {\em Aedes}.
Pour le Chikungunya, \cite{yakob_mathematical_2013}

En particulier, des modèles similaires ont été utilisés lors de l'épidémie de Chikungunya à la Réunion en 2007, et lors des épidémies de Zika






\subsection{Approches alternatives aux modèles de type Ross-Macdonald}
\label{sec:sir}

\subsubsection{Modèles mécanistes}

Une autre classe de modèles de maladies vectorielles s'inspire directement des théories et modèles utilisées pour les pathogènes transmis directement.
Ces modèles, construits autour des données d'incidence fournies par les systèmes de surveillance, se distinguent principalement par l'absence de modélisation explicite des populations de vecteurs, considérant alors seulement la transmission entre hôtes.

\subsubsection{Modèles non-mécanistes}

Exponential growth (Wallinga J, Lipsitch M. How generation intervals shape the
relationship between growth rates and reproductive numbers.
Proc Biol Sci. 2007;274(1609):599-604. http://dx.doi.
org/10.1098/rspb.2006.3754, 
Cauchemez, Nishiura)

Autoregression models ARIMA SARIMA

ecological niche modelling

neural networks thailand



%Deux principales approches ont été proposées pour modéliser de façon mécaniste la transmission des maladies vectorielles dans les populations, qui peuvent se distinguer selon la façon d'intégrer l'action du vecteur. 
%Une première classe de modèles se caractérise par l'intégration de l'action du vecteur de façon explicite, considérant le cycle complet de transmission d'un pathogène entre populations d'hôtes et populations de vecteurs, et les relations entre ces entités. 
%Ce type de méthode trouve son origine dans les modèles de type {\em Ross-Macdonald}, initialement développés pour étudier la transmission du paludisme.
%Une deuxième approche consiste à intégrer l'action du vecteur de façon implicite, considérant alors seulement la transmission entre hôtes. 
%Ce type de modèles ressemble à ceux développés pour étudier les maladies transmises directement entre humains, par exemple par voie respiratoire pour la grippe. 
%Dans ce cas, le vecteur n'est considéré que comme un lien existant entre un cas infectieux et un cas secondaire dont il est à l'origine.


 
Considérant 


Nous terminerons cette présentation des modèles de maladies vectorielles par la présentation d'une comparaison d'un modèle de type Ross-Macdonald et d'un modèle de type SIR \cite{pandey2013comparing}. 


On retrouve donc trois compartiments humains, le nombre de susceptibles $S$, d'infectieux $I$ et d'immunisés $R$, et deux compartiments de moustiques, le nombre de susceptibles $X$ et le nombre d'infectieux $Z$ (les notations ont été modifiées pour correspondre à la table \ref{table:sign}). 
Le modèle VH est gouverné par le système d'équations différentielles suivant :

\begin{align}
\frac{dS}{dt} &= fN - mab\frac{Z}{M}S - fS \\ \nonumber
\frac{dI}{dt} &= mab\frac{Z}{M}S - rI - fI \\ \nonumber
\frac{dR}{dt} &= rI - fR \\  \nonumber
\frac{dX}{dt} &= gM - ac\frac{I}{N}X - gX \\ \nonumber
\frac{dZ}{dt} &= ac\frac{I}{N}X - gZ
\end{align}

\noindent Ce modèle fait l'hypothèse que les moustiques restent infectieux jusqu'à leur décès, et qu'il n'existe pas de mortalité additionnelle due à l'infection chez les hôtes comme chez les vecteurs.
Les données de surveillance mensuelle du nombre de cas de dengue en Thaïlande de janvier 1984 à mars 1985 (Fig. \ref{fig:pandey_figure_incidence}) sont utilisées pour l'estimation des paramètres par inférence Bayésienne, en utilisant des méthodes de Monte Carlo par chaîne de Markov (MCMC, cf. \S \ref{sec:infbay}).
Pour ce faire, un compartiment est ajouté représentant le nombre cumulé de cas rapportés par le système de surveillance, grâce à l'introduction du paramètre $\rho$ :

%\begin{align}
%\frac{dO}{dt} = \rho mab\frac{Z}{M}S
%\end{align}


\begin{figure}[t]
	\centering
	\includegraphics[width=8cm]{Figures/pandey_fig_incidence.PNG}
	\caption{Incidence mensuelle de dengue en Thaïlande entre janvier 1984 et mars 1985 (source : Pandey et al, 2013)}
	\label{fig:pandey_figure_incidence}
\end{figure}


Les auteurs font l'hypothèse que le taux de décès est égal au taux de naissances dans les deux populations, et fixent la mortalité humaine à $f=1/69$ (correspondant à une durée de vie moyenne de 69 ans dans ce pays).
Les paramètres estimés sont donc : le taux de mortalité chez les vecteurs $g$, deux paramètres composites mesurant la transmission des vecteurs vers les hôtes ($\beta_H=mab$) et des hôtes vers les vecteurs ($\beta_V=ac$) puisque les paramètres $a$, $b$, $c$ et $m$ n'apparaissent que multipliés entre eux et ne sont donc pas identifiables séparément, le taux de guérison des hôtes $r$, et la proportion des cas infectés rapportés par le système de surveillance $\rho$.
De plus, on estime la proportion initiale d'humains résistants $R(0)/N$ et de moustiques infectieux $Z(0)/M$.
On note que $\mathcal{R}_0$ n'est pas un paramètre, mais plutôt une fonction de paramètres retrouvée grâce à l'analyse du système par la méthode de la matrice de génération suivante :
\begin{equation}
\mathcal{R}_0 = \frac{ma^2bc}{g(f+r)}
\end{equation}
Les résultats du modèle VH sont présentés dans la table \ref{table:pandeyres}, en comparaison avec les résultats issus d'une approche implicite.

\begin{table}[h]
\centering
\caption{Résultats des modèles vecteurs-hôtes (VH) et SIR dans Pandey et al, 2013. \vspace{.5em}}
\label{table:pandeyres}
\begin{tabular}{lllll}
\hline 
Paramètre & Unité & Distribution {\em a priori} &\multicolumn{2}{c}{Distribution postérieure}\\ 
&&& Modèle VH & Modèle SIR \\
\hline
$\beta_H$ 	& j$^{-1}$ & $\mathcal{U}(0,1)$ 		& 0.05 (0.01; 0.22) & -- \\
$\beta_V$ 	& j$^{-1}$ & $\mathcal{U}(0.1,2)$ 		& 0.49 (0.13; 1.68) & -- \\
$\beta$ 	& j$^{-1}$ & $\mathcal{U}(0,10)$ 		& 0.49 (0.28; 0.94) & 0.32 (0.19; 0.58) \\
$r$ 		& j$^{-1}$ & $\mathcal{U}(0.1,0.6)$ 		& 0.25 (0.15; 0.44) & 0.27 (0.13; 0.53)\\
$\rho$ 		& j$^{-1}$ & $\mathcal{U}(0,0.1)$ 		& 0.002 (0.001; 0.009) & 0.006 (0.002; 0.035) \\
$g$ 		& j$^{-1}$ & $\mathcal{U}(0.01,0.1)$ 		& 0.05 (0.04; 0.08) & -- \\
$R(0)/N$ 	& \% & $\mathcal{U}(0,1)$ 		& 0.2 (0; 13) & 0.2 (0; 14)\\
$Z(0)/M$ 	& \% & $\mathcal{U}(0,1)$ 		& 0.05 (0; 0.6) & -- \\
$\mathcal{R}_0$ & -- & --		 		& 1.97 (1.36; 3.21) & 1.20 (1.05; 1.52)\\
\hline 
\end{tabular} 
\end{table}







\section{Discrétisation}

\subsection{Le modèle TSIR}

On s'intéresse principalement aux dates d'incidence.

\subsection{Reconstruction mécaniste du temps de génération}

\subsection{Facteurs influençant le niveau de transmission}

\section{Approche multi-niveaux}

\subsection{Différences avec les approches classiques}

Construire un modèle complet avec SIR et moustiques (Kucharski), choisir $R_0$, EIP etc.

Simuler des épidémies depuis ce modèle.

Récupérer le temps de génération, essayer de récupérer les paramètres avec un TSIR.

\section{Inférence Bayésienne et méthodes de type MCMC}
\label{sec:infbay}

\subsection{Mieux prendre en compte l'incertitude}

\subsection{Efficacité computationnelle : Stan et l'algorithme NUTS}

\subsection{Mise en commun de plusieurs sources d'information}

\subsection{Intégrer l'information {\em a priori}}



