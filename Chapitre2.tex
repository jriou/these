\chapter{Techniques et concepts en modélisation des épidémies vectorielles émergentes}
\chaptermark{Modéliser}

\section{Les modèles de maladies vectorielles}

\subsection{SIR}

\subsection{Ronald Ross et le paludisme}

\subsection{Extensions : immunité, saisonalité, écologie, spatialisation}

\section{Contourner le manque d'information sur les vecteurs : discretisation et temps de génération}

\subsection{Discrétisation du modèle SIR}

\subsection{Reconstruire le temps de génération}

\subsection{Différences avec les approches classiques}

Construire un modèle complet avec SIR et moustiques (Kucharski), choisir $R_0$, EIP etc.

Simuler des épidémies depuis ce modèle.

Récupérer le temps de génération, essayer de récupérer les paramètres avec un TSIR.

\section{Apports des statistiques Bayésiennes}

\subsection{Mieux prendre en compte les incertitudes}

\subsection{Intégrer l'information {\em a priori}}

\subsection{Mise en commun de plusieurs sources d'information}

\subsection{Efficacité computationelle : l'algorithme HMC}