\chapter{Les invasions mondiales des moustiques du genre {\em Aedes} et leurs conséquences}
\chaptermark{}

\section[Origines, adaptations et invasions]{Origines, adaptations et invasions}

\subsection{Les moustiques du genre {\em Aedes} : éléments entomologiques}


{\em Aedes} est un genre appartenant à l'ordre des {\em Diptera} (mouches), à la famille des {\em Culicidae} (moustiques) et à la sous-famille des {\em Culicinae} (par opposition aux {\em Anophelinae} ou anophèles).
Le nom vient du grec ancien  $\alpha \eta \delta \eta \varsigma$ (aedes), signifiant "déplaisant" ou "dégoûtant".
Ce genre a été décrit brièvement pour la première fois par l'entomologiste allemand Johann Wilhelm Meigen en 1818 \cite{meigen1818}.
Par la suite, de nombreuses espèces très diverses lui ont été attribuées, ce qui a engendré une grande hétérogénéité à l'intérieur du genre.
En 2000, John Reinert a proposé une reclassification basée sur les caractéristiques des organes sexuels masculins et féminins, déplaçant de nombreuses espèces du genre {\em Aedes} au genre {\em Ochlerotatus}, tous deux regroupés dans la tribu {\em Aedini} \cite{reinert2000new}.
Selon cette classification, le genre {\em Aedes} regroupe 23 sous-genres et 263 espèces, en particulier {\em Ae. aegypti} et {\em Ae. albopictus}, vecteurs de nombreuses maladies humaines et animales. 
Plus récemment, d'autres classifications ont été proposées prenant en compte les résultats d'études phylogénétiques et qui, en particulier, amèneraient à considérer les espèces {\em Ae. aegypti} et {\em Ae. albopictus} comme faisant partie du genre {\em Stegomyia} (devenant ainsi respectivement {\em Stg. aegypti} et {\em Stg. albopictus}) \cite{reinert2004phylogeny}.
Toutefois, cette nouvelle dénomination ne faisant pas consensus au sein des entomologistes \cite{polaszek2006two} et n'étant pas d'usage courant en médecine, nous continuerons dans ce travail à utiliser les noms classiques {\em Ae. aegypti} et {\em Ae. albopictus}.


\begin{figure}[t]
	\centering
	\includegraphics[width=9cm]{Figures/aedes_aegypti_wellcome_collection.jpg}
	\caption{{\em Aedes aegypti femelle} (source: Wellcome collection)}
	\label{fig:aedesvexans}
\end{figure}

\subsubsection{Morphologie et cycle de vie}
Les {\em Aedes} adultes se distinguent des autres moustiques par leur abdomen long et étroit et par différents motifs noir et blanc sur le thorax, l'abdomen et les pattes (Fig. \ref{fig:aedesvexans}).
Comme les autres mouches, les {\em Aedes} possèdent un cycle de vie passant par quatre stades de développement : \oe uf, larve, pupe et adulte (Fig. \ref{fig:mosquitolifecyle}) \cite{christophers1960aedes}.
Contrairement à d'autres espèces de moustiques, les femelles adultes déposent leurs \oe ufs non pas directement dans l'eau mais sur un support susceptible d'être inondé, à proximité du bord d'une réserve d'eau naturelle ou artificielle, comme les rivages de marais, les arbres creux, les pots de fleurs, les récipients en plastique, ou encore les pneus usagers (Fig. \ref{fig:oeufsbord}).
Les \oe eufs sont typiquement noirs, gluants, et mesurent environ 0.5 mm.
Ils peuvent survivre plusieurs mois dans un environnement froid ou sec (et résistent donc à la dessication).
La submersion des \oe ufs à l'occasion de précipitations ou d'un remplissage artificiel déclenche ensuite l'éclosion.
Les larves vivent dans l'eau, juste sous la surface car elles doivent respirer à l'air libre par de courtes trompes, et se nourrissent de micro-organismes. 
Dans des conditions adaptées de qualité de l'eau, de température, de présence de nourriture et d'absence de prédateurs, elles se développent en passant par quatre stades appelés {\em instars} jusqu'à leur transformation en pupes.
Au stade pupal, le moustique ne se nourrit pas et achève seulement sa transformation en adulte.
Ce processus de développement peut être complété en quelques jours à quelques mois suivant l'espèce et les conditions environnementales, en particulier la température.


\begin{figure}[h]
	\centering
	\includegraphics[width=10cm]{Figures/mosquitolifecyle.pdf}
	\caption{Cycle de vie d'{\em Aedes aegypti} (source: d'après CDC)}
	\label{fig:mosquitolifecyle}
\end{figure}

\subsubsection{Hématophagie}
Dans les premiers jours suivant leur émergence, les moustiques adultes s'accouplent, puis les femelles prennent leur premier repas de sang. 
La recherche d'un hôte par l'adulte femelle se déroule préférentiellement durant la journée, et s'appuie sur de nombreux sens dont la vision, l'audition, l'odorat et le toucher. 
Les comportements, les cibles préférentielles humaines ou animales, ainsi que le degré de spécialisation dans la piqûre d'un hôte en particulier varient selon les espèces et les environnements \cite{bowen1991sensory}.
Lors de la piqûre, la salive du moustique est injectée dans l'hôte afin de fluidifier le sang. 
C'est à cette étape qu'un pathogène peut être transmis si il est présent en quantité suffisante dans la salive.
Ainsi, la compétence vectorielle, c'est à dire la capacité à transmettre un virus, dépend largement de la capacité de ce virus à se reproduire dans les glandes salivaires du moustique, et donc de l'adaptation du virus à une espèce particulière.
Ces repas de sang sont seulement nécessaire pour la maturation des \oe ufs, et le reste du temps les moustiques se nourrissent de nectars et de fruits.
Les femelles entrent ainsi dès leur émergence dans un {\em cycle gonotrophique} qui fait se succéder (1) recherche et piqûre d'un ou plusieurs hôtes par la femelle à jeun, (2) digestion du sang et maturation des \oe ufs, et (3) recherche d'un lieu de ponte et oviposition.
Là encore, la durée des différentes étapes de ce cycle varie selon l'espèce et la température.


\begin{figure}[h]
	\centering
	\includegraphics[width=10cm]{Figures/Ochlerotatus_japonicus_eggs_c2006_Omar_Fahmy.jpg}
	\caption{Oeufs d'{\em Ochlerotatus japonicus} déposés au bord d'un récipient (source: Omar Fahmy, 2006).}
	\label{fig:oeufsbord}
\end{figure}

\subsubsection{Compétence vectorielle}
Certaines espèces d'{\em Aedes} sont des vecteurs de maladies humaines très importantes comme la dengue, la fièvre jaune, le chikungunya et le Zika, mais aussi de très nombreuses autres maladies moins fréquentes incluant la fièvre du Nil occidental, la fièvre de la vallée du Rift, l'encéphalite équine de l'Est, la filariose lymphatique, ou encore les fièvres dites Mayaro, Usutu et Ross River.
Pour une large part de ces maladies, particulièrement les plus importantes, {\em Ae. aegypti} est le vecteur principal dans les zones où il est présent, c'est à dire la plupart des zones tropicales et subtropicales.
D'autres espèces d'{\em Aedes} sont parfois aussi capables d'agir comme vecteur de certaines de ces maladies, seules ou en complément d'{\em Ae. aegypti}, comme {\em Ae. albopictus}, originaire d'Asie et présent dans les zones plus tempérées d'Europe et d'Amérique du Nord, mais aussi d'autres espèces comme {\em Ae. polynesiensis} en Polynésie française et {\em Ae. scutellaris} en Asie du Sud-Est.

 
\subsection{La domestication d'{\em Aedes aegypti}}

De nos jours, {\em Ae. aegypti} est présent dans la plupart des zones tropicales et subtropicales du monde : en Afrique Subsaharienne, du nord de l'Argentine à la moitié sud des États-Unis, en Océanie, en Asie du Sud-Est et dans le sous-continent Indien (Fig. \ref{fig:repartitionaedesaegypti}) \cite{kraemer2015global}.
Il s'agit en grande majorité d'une espèce dite {\em domestique}, particulièrement active en milieu urbain.
Dans ce contexte, la domestication désigne l'adaptation d'un moustique sauvage à la vie à proximité des humains dans une relation commensale : piqûre préférentielle des humains par rapport aux animaux, survie et reproduction dans les installations humaines, avec une oviposition pouvant avoir lieu dans des récipients artificiels comme les pots de plantes, réservoirs d'eau, fosses septiques, pneus usagers ou autres déchets pouvant recueillir l'eau de pluie  \cite{christophers1960aedes}.
Cette adaptation procure des avantages majeurs à l'espèce, lui permettant d'avoir accès à un nombre  d'hôtes important et de ne plus dépendre de la pluie pour sa reproduction.


\begin{figure}[h]
	\centering
	\includegraphics[width=15cm]{Figures/repartitionaedesaegypti.jpg}
	\caption{Probabilité de présence d'{\em Ae. aegypti} dans le monde (source : Kraemer et al, 2015)}
	\label{fig:repartitionaedesaegypti}
\end{figure}


\subsubsection{Premières descriptions et identification comme vecteur de maladies humaines}

{\em Ae. aegypti} a été décrit pour la première fois en 1757 par les naturalistes suédois Fredric Hasselquist et Carl von Linné lors d'un voyage au Levant \cite{iterpalestinum1757} (d'où le mot {\em Linnaeus} ou l'abréviation {\em L.} qui figure parfois après le nom de l'espèce). Il fut d'abord nommé {\em Culex aegypti}, {\em Culex} étant le nom générique donné aux moustiques avant la formalisation des classifications.
Il a ensuite été identifié comme vecteur de la fièvre jaune à Cuba en 1900 par Walter Reed et James Carroll, médecins militaires américains, suite à une expérience consistant à faire piquer des volontaires humains (y compris eux-mêmes) par des moustiques ayant préalablement piqué des cas sévères de fièvre jaune (Fig. \ref{fig:yellowjack}) \cite{tabachnick1991evolutionary}.
De cette période est resté son nom commun en anglais de \guillemotleft yellow fever mosquito\guillemotright \;.
Dans les années suivantes, le rôle d'{\em Ae. aegypti} dans la transmission de la dengue fut mise en évidence en Australie, là aussi grâce à des expériences d'inoculation de volontaires.
Ces avancées, suivant de près la découverte du rôle des anophèles dans la transmission du paludisme par Ronald Ross en 1897, ont été le prélude à un vaste effort de recherche sur la physiologie des vecteurs de maladies, qui a donné le champ de l'entomologie médicale.
Par la suite a été mise en évidence la compétence vectorielle d'{\em Ae. aegypti} pour d'autres arboviroses importantes pour la santé publique mondiale comme le virus Zika et le virus du Chikungunya.

\begin{figure}[h]
	\centering
	\includegraphics[width=12cm]{Figures/yellowjack.jpg}
	\caption{Scène du film {\em Yellow jack} (1938) racontant l'histoire de la découverte du rôle d'{\em Ae. aegypti} dans la transmission de la fièvre jaune (source: MGM)}
	\label{fig:yellowjack}
\end{figure}



\subsubsection{Extension et domestication}

L'ubiquité de distribution, le nombre de maladies transmises et donc l'ampleur de l'impact sur la santé publique mondiale d'{\em Ae. aegypti} peut paraître surprenant, mais cet état de fait peut être relié à la relation particulière qui existe entre ce moustique et l'espèce humaine.
Il faut pour cela revenir à l'ancêtre commun aux {\em Ae. aegypti} aujourd'hui répartis sur toute la planète, qui vivait probablement dans les forêts d'Afrique à environ 6000 AEC (Fig. \ref{fig:aegyptisnp}) \cite{brown2014human}.
Une forme ancestrale du moustique subsiste d'ailleurs toujours dans certaines forêts d'Afrique subsaharienne, parfois classée comme une sous-espèce nommée {\em Ae. aegypti formosus} (abrégé en {\em Aaf}, par opposition à la forme domestiquée nommée alors {\em Ae. aegypti aegypti}, ou {\em Aaa}) \cite{powell2013history}.
Il est intéressant de constater que {\em Aaf} présente un comportement radicalement différent de {\em Aaa} et adapté au milieu sylvestre : la femelle dépose ses \oe ufs dans les souches et les trous des arbres et pique préférentiellement les hôtes non-humains.
D'autre part, l'analyse du polymorphisme génétique suggère qu'{\em Ae. aegypti} a atteint les Amériques depuis la côte ouest de l'Afrique entre le XVI\textsuperscript{ème} et le XVII\textsuperscript{ème} siècle, probablement sur les bateaux des européens pratiquant le commerce triangulaire.
Cette hypothèse est cohérente avec les données historiques qui datent la première épidémie de fièvre jaune dans le nouveau monde à 1648, dans la péninsule du Yucatan.
L'extension s'est ensuite poursuivie vers l'ouest, des Amériques vers l'Océanie puis l'Asie, et non pas directement depuis l'Afrique vers l'Asie.
Cela implique la domestication de {\em Aaa} s'est produite en Afrique avant le XVII\textsuperscript{ème} siècle, car la survie au long cours sur les bateaux humains n'aurait pas été possible sans cette adaptation.
Walter Tabachnick fait l'hypothèse d'un lien avec l'avènement du néolithique en Afrique du Nord et l'assèchement progressif du Sahara entre 6000 et 4000 AEC, laissant comme seules sources d'eau les réserves artificielles \cite{tabachnick1991evolutionary}.
A partir du milieu du XX\textsuperscript{ème} siècle, l'augmentation des populations humaines et l'urbanisation a favorisé l'extension et l'accroissement des populations de {\em Aaa}.

\begin{figure}[t]
	\centering
	\includegraphics[width=15cm]{Figures/aegyptisnp.jpg}
	\caption{Analyse du polymorphisme génétique (SNP) de spécimens contemporains d'{\em Ae. aegypti} selon deux méthodes (\guillemotleft neighbor-joining\guillemotright \;  à gauche et \guillemotleft Bayesian population tree\guillemotright \; à droite). La couleur correspond à l'origine des spéciments: en rouge l'Afrique de l'Est, en rose d'Afrique de l'Ouest, en bleu des Amériques, et bleu clair d'Asie-Pacifique (source: Brown et al, 2014)}
	\label{fig:aegyptisnp}
\end{figure}


\subsubsection{Une double adaptation}

Les différences entre {\em Aaa} et {\em Aaf} ne s'arrêtent pas aux comportements, mais concernent aussi la compétence vectorielle. 
Ainsi, il a été démontré que la capacité à transmettre la fièvre jaune et la dengue est plus faible dans les populations {\em Aaf} que les populations {\em Aaa} \cite{black2002flavivirus}.
A l'inverse, une autre étude indique que la compétence vectorielle de {\em Aaa} est plus faible pour une souche sylvatique \guillemotleft sauvage\guillemotright \;\, de dengue que pour une souche causant des épidémies parmi les populations humaines \cite{moncayo2004dengue}.
Ce résultat suggère que ce phénomène de domestication d'{Ae. aegypti} a été accompagné par une augmentation de la capacité à transmettre certains virus humains.
Cela pourrait s'expliquer par une pression de sélection favorisant l'adaptation d'un virus à ceux des moustiques qui ont une intense activité de piqûre d'hôtes humains, augmentant les chances de transmission inter-humaine \cite{powell2013history}.
On est donc en présence d'une double adaptation avec d'une part des moustiques adaptant leur comportement à la présence humaine et multipliant les contacts potentiellement infectieux, et d'autre part des virus qui s'adaptent pour être plus facilement transmis. 
Cela forme un ensemble particulièrement puissant, exerçant une intense pression en faveur de l'émergence ou de la réémergence de maladies épidémiques humaines, et avec pour effets les récentes épidémies mondiales de Chikungunya et de Zika.



\subsection{L'invasion d'{\em Aedes albopictus} }


{\em Ae. albopictus}, communément appelé \guillemotleft moustique tigre\guillemotright \;, est une des espèces les plus invasives de l'histoire mondiale. 
Originaire d'Asie du Sud-Est, il est aujourd'hui présent dans de nombreuses zones tempérées et subtropicales du monde (Fig. \ref{fig:repartitionaedesalbopictus}). 
Une des différences majeurs avec {\em Ae. aegypti} est qu'il est capable de s'adapter à des températures plus froides.
Si sa morphologie est très proche de celle d'{\em Ae. aegypti}, {\em Ae. albopictus} se différencie par une teinte générale plus sombre et une unique bande blanche dorsale longitudinale.
L'extension d'{\em Ae. albopictus} et ses potentielles répercussions sur la santé publique ont été longtemps minimisées, de nombreux spécialistes considérant qu'il avait une faible capacité de transmission de maladies humaines \cite{paupy2009aedes}. 
Il a été démontré par la suite qu'{\em Ae. albopictus} est en fait capable de transmettre 26 virus humains, et joue un rôle important dans la transmission de la dengue et du chikungunya.


\begin{figure}[t]
	\centering
	\includegraphics[width=15cm]{Figures/repartitionaedesalbopictus.jpg}
	\caption{Probabilité de présence d'{\em Ae. albopictus} dans le monde (source : Kraemer et al, 2015)}
	\label{fig:repartitionaedesalbopictus}
\end{figure}


\subsubsection{Écologie et avantages compétitifs}

L'espèce a été décrite pour la première fois en 1894 par Frederick Arthur Askew Skuse dans le golfe du Bengale sous le nom de {\em Culex albopictus} avant d'être classée dans le genre {\em Aedes} (là encore, son nom est parfois suivi du nom de son découvreur, {\em Skuse}) \cite{skuse1894banded}. 
De façon générale, {\em Ae. albopictus} est remarquable par sa plasticité. 
Il est ainsi capable de se reproduire et de survivre à des températures bien plus basses qu'{\em Ae. aegypti}, jusqu'à 10$^\circ$C au Japon et à la Réunion, et jusqu'à -5$^\circ$C aux États-Unis \cite{paupy2009aedes}. 
En dessous de ces limites, les \oe ufs sont capables d'entrer en hibernation pour des périodes prolongées.
Originellement sylvestre, {\em Ae. albopictus} s'est progressivement adapté aux modifications de l'environnement induites par les humains, mais, contrairement à la forme domestiquée d'{\em Ae. aegypti}, n'a pas développé de dépendance à la vie à proximité des humains.
Son oviposition se fait préférentiellement dans des lieux naturels, mais il est aussi capable comme {\em Ae. aegypti} d'utiliser des récipients artificiels (Fig. \ref{fig:larval_breeding}).
{\em Ae. albopictus} a aussi un comportement opportuniste vis à vis des hôtes et peut piquer aussi bien les humains, les autres mammifères, les oiseaux ou même certains reptiles et amphibiens, même s'il semble avoir une préférence pour les humains.
Il peut être retrouvé aussi bien dans les zones péri-urbaines que rurales ou forestières, mais n'est abondant que dans certaines centres urbains très boisés comme Singapour, Tokyo ou Rome \cite{paupy2009aedes}.
Dans les zones où {\em Ae. albopictus} et {\em Ae. aegypti} cohabitent, on observe une compétition entre les deux espèces, qui tourne d'ailleurs généralement à l'avantage de ce dernier.
En effet, {\em Ae. albopictus} présente en effet des caractéristiques qui lui donnent un avantage compétitif sur d'autres espèces concurrentes sous certaines conditions.
Les mécanismes en jeu sont principalement la compétition pour les ressources nécessaires au développement larvaire, mais d'autres ont été suggérés comme l'apport de parasites (), les interférences chimiques (les déjections des larves d'{\em Ae. albopictus} limiteraient le développement ultérieur des larves d'{\em Ae. aegypti}), et les interférences d'accouplement (les mâles {\em Ae. albopictus} recherchant plus agressivement l'accouplement avec des femelles {\em Ae. aegypti} que l'inverse, causant une baisse de la fécondité) \cite{juliano2005ecology}.
L'installation d'{\em Ae. albopictus} a ainsi été associée au déclin des populations d'{\em Ae. aegypti} en Amérique du Nord et au Brésil \cite{paupy2009aedes}.


\begin{figure}[t]
	\centering
	\includegraphics[width=15cm]{Figures/larval_breeding_sites_cDidier_Fontenille_IRD.pdf}
	\caption{Exemple de site de développement larvaire naturel (A) ou artificiel (B) d'{\em Aedes albopictus} (source : Didier Fontenille, IRD)}
	\label{fig:larval_breeding}
\end{figure}

\subsubsection{L'extension fulgurante d'{\em Ae. albopictus}}

Les avantages compétitifs d'{\em Ae. albopictus} permettent d'expliquer sa prééminence sur les espèces locales lorsqu'il est introduit dans un nouveau milieu présentant des conditions favorables.
De fait, de multiples introductions ont eu lieu dans le monde entier à partir des années 1960 avec la multiplication des échanges commerciaux.
En particulier, la dispersion d'\oe ufs ou de larves par le commerce international de pneus usagers semble avoir joué un rôle prépondérant \cite{reiter1998aedes}.
Originaire d'Asie, {\em Ae. albopictus} a ainsi été introduit et est devenu prédominant dans de très nombreuses zones du monde en quelques dizaines d'années seulement.
Il est aujourd'hui établit sur le continent Américain des États-Unis à l'Argentine, dans de nombreuses îles d'Océanie (mais pas en Polynésie française, où {\em Ae. aegypti} et {\em Ae. polynensis} dominent), en Afrique australe et centrale, en Europe méditerranéenne (dans les Balkans, en Italie, en Espagne, dans le sud de la France) et poursuit sa progression vers l'Europe centrale (Nord de la France, Suisse, Allemagne, Pays-Bas) (Fig. \ref{fig:repartitionaedesalbopictus}) \cite{kraemer2015global}
Cette extension fulgurante peut être mise en parallèle avec celle, plus ancienne et plus lente d'{\em Ae. aegypti}, suivant les navigateurs des XVI\textsuperscript{ème} et XVII\textsuperscript{ème} siècles.


\section{Émergences et réémergences}

La population humaine ainsi que les échanges internationaux ont connu une croissance sans précédent au cours du XX\textsuperscript{ème} siècle.
Alors que l'urbanisation est associée à l'augmentation des populations d'{\em Ae. aegypti}, l'augmentation des échanges a favorisé l'extension des zones d'activité d'{\em Ae. albopictus}. 
Ces phénomènes ont entraîné une multiplication des contacts entre vecteurs et populations humaines, qui s'est traduite par la résurgence d'épidémies de maladies anciennes comme la dengue et la fièvre jaune, mais aussi par la dissémination de virus initialement circonscrits à des zones géographiques limitées, comme les virus du chikungunya ou du Zika.
Dans cette partie, nous passerons en revue l'épidémiologie des principales maladies transmises par les moustiques du genre {\em Aedes}.

\subsection{Fièvre jaune}

%épidémie brésil 2018


\subsection{Dengue}
\label{sec:dengue}

Le virus de la dengue (généralement abrégé en DENV) est un virus à ARN appartenant au genre {\em Flavivirus} et qui comprend cinq sérotypes différents (le dernier, DENV-5, ayant été identifié en 2013).
Il semble que ces sérotypes aient émergé il y a environ mille ans, mais n'aient établi de transmission endémique chez l'humain que depuis quelques centaines d'années \cite{holmes2003origin}.
Les premières épidémies de maladies cliniquement semblables à la dengue ont été décrites vers la fin du XVIII\textsuperscript{ème} siècle en Asie et aux Amériques.
A la fin du XIX\textsuperscript{ème} siècle, la dengue était présente dans de nombreuses zones tropicales et subtropicales, et le virus a pu être isolé pour la première fois au Japon en 1943 (DENV-1) et à Hawaï en 1945 (DENV-2).
Par la suite, la maladie est devenue endémique dans les pays d'Asie du sud et du sud-est et d'Amérique centrale du sud (Fig. \ref{fig:denvworld}), avec une augmentation importante du nombre global de cas rapportés à partir des années 1990 pour atteindre 60 millions d'infections symptomatiques et 10 000 décès par an en 2013 \cite{messina2014global,stanaway2016global}.

L'infection par DENV est asymptomatique dans 75 à 90\% des cas, mais peut aussi se traduire par une fièvre indifférenciée et spontanément résolutive en 3 à 7 jours, parfois accompagnée de myalgies, d'arthralgies et d'éruptions cutanées.
Dans de rares cas, l'infection peut causer une fièvre hémorragique avec complication possible en syndrome de choc, qui peut être fatal dans 1 à 13\% des cas. La survenue de cas graves étant plus fréquente si le sujet a été infecté par un autre sérotype dans le passé \cite{simmons2012dengue}.
L'infection par un sérotype donné confère une immunité temporaire contre tous les sérotypes, et une immunité à long terme contre ce sérotype seulement.
Cette caractéristique fait que les dynamiques de transmission et de dissémination dans les populations sont très complexes.
Les tests 

%
%Dengue is the major arbovirose (arthropod-borne virus) in the
%world and a leading cause of hospitalization and death among
%children in Asia [13,72,118]. It is especially prevalent in tropical
%regions, where the primary vector Aedes aegypti thrives. Although
%Aedes albopictus was shown to be less efficient for dengue
%transmission than Aedes aegypti, its role was clearly established in
%a few dengue outbreaks in areas free from the primary vector (e.g.
%in Japan in 1942 and more recently in Hawaii (2001)

\begin{figure}[t]
	\centering
	\includegraphics[width=15cm]{Figures/DENVworld_messina.jpg}
	\caption{Extension géographique des zones de circulation des sérotypes de DENV entre 1943 et 2013 (source : Messina et al, 2014)}
	\label{fig:denvworld}
\end{figure}




\subsection{Filariose lymphatique}

\subsection{Chikungunya}
 %\cite{vega2014high}

%Emergence often involves adaptation to new hosts. Pasteur argued that attenuated forms of virulent parasites already exist in populations and that their 'virulence can be progressively reinforced' if the environmental conditions are adequate [9, 12]. The Chikungunya virus outbreak that occurred in La Réunion Island in 2005 to 2006 illustrates how such 'reinforcement' may occur. At the end of 2005, there was a first limited outbreak that caused a few thousand cases. In 2006, it was followed by a huge outbreak with hundreds of thousands of cases. The main reason for the size difference was that in 2006, most viruses bore a key mutation in position 226 of the E1 protein, which greatly increased the vectorial capacity of Aedes albopictus (tiger) mosquitoes [24], which are known for their anthropophilic behaviour.  (alizon)
%
%Yakob
%Chikungunya is an alphavirus that infects humans through
%bites from Aedes spp. mosquitoes. Symptoms are similar to those
%of dengue fever during the acute phase and include rash and
%high fever that, in a small proportion of cases, can develop into a
%life-threatening haemorrhagic fever [1]. Additionally, joint pain
%that is frequently associated with infection can persist for over a
%year [2], and is responsible for its name which means ‘‘that
%which bends’’ in the Makonde language of Southern Tanzania
%and Northern Mozambique. In 2004, a major epidemic in
%Lamu, Kenya resulted in 13,500 cases [3]. This epidemic
%sparked a four-year period in which the virus spread through
%numerous islands of the Indian Ocean, India and parts of
%Southeast Asia [4]. Cases were imported to Europe and North
%America through returning travellers, and subsequent autoch-
%thonous transmission events occurred due to the wide geograph-
%ical distribution of the vectors [4].


\subsection{Zika}

\subsection{Le futur : Mayaro, Usutu, Ross River}
%\cite{long2011experimental}

\section{Stratégies de lutte}

\subsection{Vaccins}

\subsection{Lutte antivectorielle}

%canal du panama
%Bruce-Chwatt L (1977) Ronald Ross, William Gorgas, and Malaria Eradication. Am J Trop Med Hyg 26: 1071–1079.L. Bruce-Chwatt1977Ronald Ross, William Gorgas, and Malaria Eradication.Am J Trop Med Hyg2610711079
%
%éradication espagne europe aedes aegypti  (Eritja et al. 2005)
%
%Unlike A. aegypti, which has developed resistance to multiple insecticides worldwide, A. albopictus has shown an unexpectedly low level of resistance to pesticides [63], with the exception of populations in Thailand, where a decreased susceptibility of the field-caught populations to permethrin, malathion and temephos has been reported [64]. (paupy)

%Safety and mosquitocidal efficacy of high-dose ivermectin when co-administered with dihydroartemisinin-piperaquine in Kenyan adults with uncomplicated malaria (IVERMAL): a randomised, double-blind, placebo-controlled trial 
%http://www.thelancet.com/journals/laninf/article/PIIS1473-3099(18)30163-4/fulltext