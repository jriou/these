\chapter*{Introduction}
\chaptermark{Introduction}
\addcontentsline{toc}{chapter}{Introduction}

%Le genre {\em Aedes} regroupe de nombreuses espèces de moustiques, vecteurs de nombreuses maladies humaines et animales. 
%Parmi celles-ci, {\em Ae. aegypti} et {\em Ae. albopictus} ont une influence particulièrement importante sur la santé publique globale, du fait de leur ubiquité dans les zones tropicales et semi-tempérées.
%L'abondance de ces moustiques à proximité des centres de population humaine
%
%L'existence de fortes concentrations de moustiques du genre {\em Aedes} vivant à proximité de centres de population humaine dans les zones tropicales et tempérées exerce une pression de sélection favorisant l'émergence de maladies humaines transmises par ces vecteurs.
%
%
%
%Nous avons passé en revue au chapitre 1 les origines de cette situation et les conséquences déjà observées.
%Malgré de récents développements dans les stratégies de lutte antivectorielle, il est probable que de nouvelles émergences vont se produire, avec des conséquences imprévisibles sur la santé des populations.
%Au delà de l'étude individuelle des épidémies passées, qui reste un sujet important, l'observation plus générale des dynamiques épidémiques caractéristiques des maladies transmises par les moustiques du genre {\em Aedes} pourrait permettre de mieux anticiper et contrôler les émergences futures.
%Avec cet objectif, nous proposons une approche innovante consistant à comparer directement plusieurs épidémies partageant le même mode de transmission et touchant les mêmes populations.
%Nous nous sommes concentrés sur l'analyse des épidémies successives de chikungunya et de Zika ayant eu lieu en Polynésie française et aux Antilles françaises entre 2013 et 2017 au moyen d'un modèle unique hiérarchique, permettant de distinguer les influences respectives de plusieurs facteurs sur les niveaux de transmission.
%Le développement de ce modèle nous a conduit à emprunter des concepts venant de plusieurs champs de l'épidémiologie et des biostatistiques, que nous passerons en revue dans ce chapitre.
