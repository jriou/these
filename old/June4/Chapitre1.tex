\chapter{L'invasion mondiale des moustiques du genre {\em Aedes} et ses conséquences}
\chaptermark{}

La situation actuelle d'ubiquité et d'abondance des moustiques {\em Ae. aegypti} et {\em Ae. albopictus}, moustiques capables de transmettre de nombreuses maladies humaines, constitue un véritable problème de santé globale. 
En effet, ces vecteurs sont liés à la résurgence ou à l'émergence de nombreuses maladies, avec une accélération significative en ce début de XXI\textsuperscript{ème} siècle.
Nous proposons dans ce premier chapitre un les connaissances actuelles sur les origines de cette situation, en prenant en compte les aspects entomologiques, historiques et démographiques.
Nous passerons aussi en revue l'épidémiologie des principales maladies transmises par les moustiques du genre {\em Aedes}, ainsi que les moyens de lutte disponibles.

\section[Origines, adaptations \& invasions]{Origines, adaptations et invasions}

\subsection{Les moustiques du genre {\em Aedes} : éléments entomologiques}

{\em Aedes} est un genre appartenant à l'ordre des {\em Diptera} (mouches), à la famille des {\em Culicidae} (moustiques) et à la sous-famille des {\em Culicinae} (par opposition aux {\em Anophelinae} ou anophèles).
Le nom vient du grec ancien  $\alpha \eta \delta \eta \varsigma$ (aedes), signifiant \guillemotleft déplaisant\guillemotright \; ou \guillemotleft dégoûtant\guillemotright.
Ce genre a été décrit brièvement pour la première fois par l'entomologiste allemand Johann Wilhelm Meigen en 1818 \cite{meigen1818}.
Par la suite, de nombreuses espèces très diverses ont été attribuées à ce genre, ce qui a engendré une grande hétérogénéité.
En 2000, John Reinert a proposé une reclassification basée sur les caractéristiques des organes sexuels masculins et féminins, déplaçant de nombreuses espèces du genre {\em Aedes} au genre {\em Ochlerotatus}, tous deux regroupés dans la tribu {\em Aedini} \cite{reinert2000new}.
Selon cette classification, le genre {\em Aedes} regroupe 23 sous-genres et 263 espèces. 
Plus récemment, d'autres classifications ont été proposées prenant en compte les résultats d'études phylogénétiques et qui, en particulier, amèneraient à considérer les espèces {\em Ae. aegypti} et {\em Ae. albopictus} comme faisant partie du genre {\em Stegomyia} (devenant ainsi respectivement {\em Stg. aegypti} et {\em Stg. albopictus}) \cite{reinert2004phylogeny}.
Toutefois, cette nouvelle dénomination ne faisant pas consensus au sein des entomologistes \cite{polaszek2006two} et n'étant pas d'usage courant dans le domaine médical, nous continuerons dans ce travail à utiliser les noms classiques {\em Ae. aegypti} et {\em Ae. albopictus}.

\begin{figure}[t]
	\centering
	\includegraphics[width=11cm]{Figures/aedes_aegypti_wellcome_collection.jpg}
	\caption{{\em Aedes aegypti femelle} (source: A.J.E. Verzi, Wellcome collection)}
	\label{fig:aedesvexans}
\end{figure}

\subsubsection{Morphologie et cycle de vie}
Les {\em Aedes} adultes se distinguent des autres moustiques par leur abdomen long et étroit et par différents motifs noir et blanc sur le thorax, l'abdomen et les pattes (Fig. \ref{fig:aedesvexans}).
Comme les autres mouches, les {\em Aedes} possèdent un cycle de vie passant par quatre stades de développement : \oe uf, larve, pupe et adulte (Fig. \ref{fig:mosquitolifecyle}) \cite{christophers1960aedes}.
Contrairement à d'autres espèces de moustiques, les femelles adultes déposent leurs \oe ufs non pas directement dans l'eau mais sur un support susceptible d'être inondé, à proximité du bord d'une réserve d'eau naturelle ou artificielle, comme les rivages de marais, les arbres creux, les pots de fleurs, les récipients en plastique, ou encore les pneus usagés (Fig. \ref{fig:oeufsbord}).
Les \oe eufs sont typiquement noirs, gluants, et mesurent environ 0.5 mm.
Ils peuvent survivre plusieurs mois dans un environnement froid ou sec (et résistent donc à la dessication).
La submersion des \oe ufs à l'occasion de précipitations ou d'un remplissage artificiel déclenche ensuite l'éclosion.
Les larves vivent dans l'eau, juste sous la surface car elles doivent respirer à l'air libre par de courtes trompes, et se nourrissent de micro-organismes. 
Dans des conditions adaptées de qualité de l'eau, de température, de présence de nourriture et d'absence de prédateurs, elles se développent en passant par quatre stades appelés {\em instars} jusqu'à leur transformation en pupes.
Au stade pupal, le moustique ne se nourrit pas et achève seulement sa transformation en adulte.
Ce processus de développement peut être complété en quelques jours à quelques mois suivant l'espèce et les conditions environnementales, en particulier la température.


\begin{figure}[h]
	\centering
	\includegraphics[width=10cm]{Figures/mosquitolifecyle.pdf}
	\caption{Cycle de vie d'{\em Aedes aegypti} (source: CDC)}
	\label{fig:mosquitolifecyle}
\end{figure}

\subsubsection{Hématophagie}
Dans les premiers jours suivant leur émergence, les moustiques adultes s'accouplent, puis les femelles prennent leur premier repas de sang. 
La recherche d'un hôte par l'adulte femelle se déroule préférentiellement durant la journée, et s'appuie sur de nombreux sens dont la vision, l'audition, l'odorat et le toucher. 
Les comportements, les cibles préférentielles humaines ou animales, ainsi que le degré de spécialisation dans la piqûre d'un hôte en particulier varient selon les espèces et les environnements \cite{bowen1991sensory}.
Lors de la piqûre, la salive du moustique est injectée dans l'hôte afin de fluidifier le sang. 
C'est à cette étape qu'un pathogène peut être transmis si il est présent en quantité suffisante dans la salive.
Ainsi, la compétence vectorielle, c'est à dire la capacité à transmettre un virus, dépend largement de la capacité de ce virus à se reproduire dans les glandes salivaires du moustique, et donc de l'adaptation du virus à une espèce particulière.
Ces repas de sang sont seulement nécessaire pour la maturation des \oe ufs, et le reste du temps les moustiques se nourrissent de nectars et de fruits.
Les femelles entrent ainsi dès leur émergence dans un {\em cycle gonotrophique} qui fait se succéder (1) recherche et piqûre d'un ou plusieurs hôtes par la femelle à jeun, (2) digestion du sang et maturation des \oe ufs, et (3) recherche d'un lieu de ponte et oviposition.
Là encore, la durée des différentes étapes de ce cycle varie selon l'espèce et la température.


\begin{figure}[h]
	\centering
	\includegraphics[width=10cm]{Figures/Ochlerotatus_japonicus_eggs_c2006_Omar_Fahmy.jpg}
	\caption{Oeufs d'{\em Ochlerotatus japonicus} déposés au bord d'un récipient (source: Omar Fahmy, 2006).}
	\label{fig:oeufsbord}
\end{figure}

\subsubsection{Compétence vectorielle}
Certaines espèces d'{\em Aedes} sont des vecteurs de maladies humaines très importantes comme la dengue, la fièvre jaune, le chikungunya et le Zika, mais aussi de très nombreuses autres maladies moins fréquentes incluant la fièvre du Nil occidental, la fièvre de la vallée du Rift, l'encéphalite équine de l'Est, la filariose lymphatique, ou encore les fièvres dites Mayaro, Usutu et Ross River.
Pour une large part de ces maladies, particulièrement les plus importantes, {\em Ae. aegypti} est le vecteur principal dans les zones où il est présent, c'est à dire la plupart des zones tropicales et subtropicales.
D'autres espèces d'{\em Aedes} sont parfois aussi capables d'agir comme vecteur de certaines de ces maladies, seules ou en complément d'{\em Ae. aegypti}, comme {\em Ae. albopictus}, originaire d'Asie et présent dans les zones plus tempérées d'Europe et d'Amérique du Nord, mais aussi d'autres espèces comme {\em Ae. polynesiensis} en Polynésie française et {\em Ae. scutellaris} en Asie du Sud-Est.

 
\subsection{La domestication d'{\em Aedes aegypti}}

De nos jours, {\em Ae. aegypti} est présent dans la plupart des zones tropicales et subtropicales du monde : en Afrique Subsaharienne, du nord de l'Argentine à la moitié sud des États-Unis, en Océanie, en Asie du Sud-Est et dans le sous-continent Indien (Fig. \ref{fig:repartitionaedesaegypti}) \cite{kraemer2015global}.
Il s'agit en grande majorité d'une espèce dite {\em domestique}, particulièrement active en milieu urbain.
Dans ce contexte, la domestication désigne l'adaptation d'un moustique sauvage à la vie à proximité des humains dans une relation commensale : piqûre préférentielle des humains par rapport aux animaux, survie et reproduction dans les installations humaines, avec une oviposition pouvant avoir lieu dans des récipients artificiels comme les pots de plantes, réservoirs d'eau, fosses septiques, pneus usagers ou autres déchets pouvant recueillir l'eau de pluie  \cite{christophers1960aedes}.
Cette adaptation procure des avantages majeurs à l'espèce, lui permettant d'avoir accès à un nombre  d'hôtes important et de ne plus dépendre entièrement de la pluie pour sa reproduction.


\begin{figure}[t]
	\centering
	\includegraphics[width=15cm]{Figures/repartitionaedesaegypti.jpg}
	\caption{Probabilité de présence d'{\em Ae. aegypti} dans le monde (source : Kraemer et coll., 2015)}
	\label{fig:repartitionaedesaegypti}
\end{figure}


\subsubsection{Premières descriptions et identification comme vecteur de maladies humaines}

{\em Ae. aegypti} a été décrit pour la première fois en 1757 par les naturalistes suédois Fredric Hasselquist et Carl von Linné lors d'un voyage au Proche-Orient \cite{iterpalestinum1757} (d'où le mot {\em Linnaeus} ou l'abréviation {\em L.} qui figure parfois après le nom de l'espèce). Il fut d'abord nommé {\em Culex aegypti}, {\em Culex} étant le nom générique donné aux moustiques avant la formalisation des classifications.
Il a ensuite été identifié comme vecteur de la fièvre jaune à Cuba en 1900 par Carlos Finlay, Walter Reed et James Carroll suite à une expérience consistant à faire piquer des volontaires humains (y compris eux-mêmes) par des moustiques ayant préalablement piqué des cas sévères de fièvre jaune (Fig. \ref{fig:yellowjack}) \cite{tabachnick1991evolutionary}.
De cette période est resté son nom commun en anglais de \guillemotleft yellow fever mosquito\guillemotright.
Dans les années suivantes, le rôle d'{\em Ae. aegypti} dans la transmission de la dengue fut mise en évidence en Australie, là aussi grâce à des expériences d'inoculation de volontaires.
Ces avancées, suivant de près la découverte du rôle des anophèles dans la transmission du paludisme par Ronald Ross en 1897, ont été le prélude à un vaste effort de recherche sur la physiologie des vecteurs de maladies, qui a donné naissance au champ de l'entomologie médicale.
Par la suite a été mise en évidence la compétence vectorielle d'{\em Ae. aegypti} pour d'autres arboviroses importantes pour la santé publique mondiale comme le virus Zika et le virus du chikungunya.

\begin{figure}[t]
	\centering
	\includegraphics[width=12cm]{Figures/yellowjack.jpg}
	\caption{Scène du film {\em Yellow jack} (1938) racontant l'histoire de la découverte du rôle d'{\em Ae. aegypti} dans la transmission de la fièvre jaune (source: MGM)}
	\label{fig:yellowjack}
\end{figure}



\subsubsection{Extension et domestication}

L'ubiquité de distribution, le nombre de maladies transmises et donc l'ampleur de l'impact sur la santé publique mondiale d'{\em Ae. aegypti} peut paraître surprenant, mais cet état de fait peut être relié à la relation particulière qui existe entre ce moustique et l'espèce humaine.
Il faut pour cela revenir à l'ancêtre commun aux {\em Ae. aegypti} aujourd'hui répartis sur toute la planète, qui vivait probablement dans les forêts d'Afrique aux environs de 6000 AEC (Fig. \ref{fig:aegyptisnp}) \cite{brown2014human}.
Une forme ancestrale du moustique a subsisté jusqu'à aujourd'hui dans certaines forêts d'Afrique subsaharienne, généralement classée comme une sous-espèce nommée {\em Ae. aegypti formosus} (abrégé en {\em Aaf}). Selon cette classification, la sous-espèce correspondant à la forme domestiquée est alors nommée {\em Ae. aegypti aegypti} (abrégé en {\em Aaa}) \cite{powell2013history}.
Il est intéressant de constater que les moustiques de type {\em Aaf} présentent un comportement radicalement différent des moustiques de type {\em Aaa}, adapté à un milieu de vie sylvestre : la femelle dépose ses \oe ufs dans les souches et les trous des arbres et pique préférentiellement les hôtes non-humains.
D'autre part, l'analyse comparative du polymorphisme génétique des individus actuels suggère qu'{\em Ae. aegypti} a atteint les Amériques depuis la côte ouest de l'Afrique entre le XVI\textsuperscript{ème} et le XVII\textsuperscript{ème} siècle, probablement sur les bateaux des européens pratiquant la traite des esclaves \cite{bryant2007out}.
Cette hypothèse est cohérente avec les données historiques qui datent la première épidémie de fièvre jaune dans le nouveau monde à 1648, dans la péninsule du Yucatan.
L'extension s'est ensuite poursuivie vers l'ouest, des Amériques vers l'Océanie puis l'Asie, et non pas directement depuis l'Afrique vers l'Asie.
A partir du milieu du XX\textsuperscript{ème} siècle, la croissance des populations humaines et l'urbanisation massive ont favorisé la poursuite de l'extension et la prolifération de la sous-espèce {\em Aaa}, aujourd'hui largement majoritaire en particulier dans les zones urbaines et péri-urbaines.
Ce scénario, appuyé par des éléments de nature historique, génétique et écologique, implique la domestication de {\em Aaa} s'est produite en Afrique avant le XVII\textsuperscript{ème} siècle, car la survie au long cours sur les bateaux humains n'aurait pas été possible sans cette adaptation.
Certains auteurs ont fait l'hypothèse d'un lien entre cette domestication et l'avènement du néolithique en Afrique du Nord entre 6000 et 4000 AEC \cite{tabachnick1991evolutionary}.

\begin{figure}[t]
	\centering
	\includegraphics[width=15cm]{Figures/aegyptisnp.jpg}
	\caption[Analyse du polymorphisme génétique d'{\em Ae. aegypti} (source: Brown et coll., 2014)]{Analyse du polymorphisme génétique (SNP) de spécimens contemporains d'{\em Ae. aegypti} selon deux méthodes (\guillemotleft neighbor-joining\guillemotright \;  à gauche et \guillemotleft Bayesian population tree\guillemotright \; à droite). La couleur correspond à l'origine des spécimens: en rouge l'Afrique de l'Est, en rose d'Afrique de l'Ouest, en bleu des Amériques, et bleu clair d'Asie-Pacifique (source: Brown et coll., 2014)}
	\label{fig:aegyptisnp}
\end{figure}


\subsubsection{Une double adaptation}

Les différences entre les sous-espèces {\em Aaa} et {\em Aaf} ne s'arrêtent pas aux comportements, mais concernent aussi la compétence vectorielle. 
Ainsi, il a été démontré que la capacité à transmettre la fièvre jaune et la dengue est plus faible dans les populations {\em Aaf} que les populations {\em Aaa} \cite{black2002flavivirus}.
A l'inverse, une autre étude indique que la compétence vectorielle de {\em Aaa} est plus faible pour une souche sylvatique \guillemotleft sauvage\guillemotright \;\, de dengue que pour une souche causant des épidémies parmi les populations humaines \cite{moncayo2004dengue}.
Ce résultat suggère que ce phénomène de domestication d'{Ae. aegypti} a été accompagné par une augmentation de la capacité à transmettre certains virus humains.
Cela pourrait s'expliquer par une pression de sélection favorisant l'adaptation d'un virus à ceux des moustiques qui ont une intense activité de piqûre d'hôtes humains, augmentant les chances de transmission inter-humaine \cite{powell2013history}.
On est donc en présence d'une double adaptation avec d'une part des moustiques adaptant leur comportement à la présence humaine et multipliant les contacts potentiellement infectieux, et d'autre part des virus qui s'adaptent pour être plus facilement transmis. 
Cela forme un ensemble particulièrement puissant, exerçant une intense pression en faveur de l'émergence ou de la réémergence de maladies épidémiques humaines, et avec pour effets les récentes épidémies mondiales de chikungunya et de Zika.



\subsection{L'invasion d'{\em Aedes albopictus} }


{\em Ae. albopictus}, communément appelé \guillemotleft moustique tigre\guillemotright \;, est une des espèces les plus invasives de l'histoire mondiale. 
Originaire d'Asie du Sud-Est, il est aujourd'hui présent dans de nombreuses zones tempérées et subtropicales du monde (Fig. \ref{fig:repartitionaedesalbopictus}). 
Une des différences majeurs avec {\em Ae. aegypti} est qu'il est capable de s'adapter à des températures plus froides.
Si sa morphologie est très proche de celle d'{\em Ae. aegypti}, {\em Ae. albopictus} se différencie par une teinte générale plus sombre et une unique bande blanche dorsale longitudinale.
L'extension d'{\em Ae. albopictus} et ses potentielles répercussions sur la santé publique ont été longtemps minimisées, de nombreux spécialistes considérant qu'il avait une faible capacité de transmission de maladies humaines \cite{paupy2009aedes}. 
Il a été démontré par la suite qu'{\em Ae. albopictus} est en fait capable de transmettre 26 virus humains, et joue un rôle important dans la transmission de la dengue et du chikungunya.


\begin{figure}[t]
	\centering
	\includegraphics[width=15cm]{Figures/repartitionaedesalbopictus.jpg}
	\caption{Probabilité de présence d'{\em Ae. albopictus} dans le monde (source : Kraemer et coll., 2015)}
	\label{fig:repartitionaedesalbopictus}
\end{figure}


\subsubsection{Écologie et avantages compétitifs}

L'espèce a été décrite pour la première fois en 1894 par Frederick Arthur Askew Skuse dans le golfe du Bengale sous le nom de {\em Culex albopictus} avant d'être classée dans le genre {\em Aedes} (là encore, son nom est parfois suivi du nom de son découvreur, {\em Skuse}) \cite{skuse1894banded}. 
De façon générale, {\em Ae. albopictus} est remarquable par sa plasticité. 
Il est ainsi capable de se reproduire et de survivre à des températures bien plus basses qu'{\em Ae. aegypti}, jusqu'à 10$^\circ$C au Japon et à la Réunion, et jusqu'à -5$^\circ$C aux États-Unis \cite{paupy2009aedes}. 
En dessous de ces limites, les \oe ufs sont capables d'entrer en hibernation pour des périodes prolongées.
Originellement sylvestre, {\em Ae. albopictus} s'est progressivement adapté aux modifications de l'environnement induites par les humains, mais, contrairement à la forme domestiquée d'{\em Ae. aegypti}, n'a pas développé de dépendance à la vie à proximité des humains.
Son oviposition se fait préférentiellement dans des lieux naturels, mais il est aussi capable comme {\em Ae. aegypti} d'utiliser des récipients artificiels (Fig. \ref{fig:larval_breeding}).
{\em Ae. albopictus} a aussi un comportement opportuniste vis à vis des hôtes et peut piquer aussi bien les humains, les autres mammifères, les oiseaux ou même certains reptiles et amphibiens.
Il peut être retrouvé aussi bien dans les zones péri-urbaines que rurales ou forestières, mais n'est abondant que dans certaines centres urbains très boisés comme Singapour, Tokyo ou Rome \cite{paupy2009aedes}.
Dans les zones où {\em Ae. albopictus} et {\em Ae. aegypti} cohabitent, on observe une compétition entre les deux espèces, qui tourne d'ailleurs généralement à l'avantage du premier.
En effet, {\em Ae. albopictus} présente des caractéristiques qui lui donnent un avantage compétitif sur d'autres espèces concurrentes sous certaines conditions.
Les mécanismes en jeu sont principalement la compétition pour les ressources nécessaires au développement larvaire, mais d'autres ont été suggérés comme l'apport de parasites, les interférences chimiques (les déjections des larves d'{\em Ae. albopictus} limiteraient le développement ultérieur des larves d'{\em Ae. aegypti}), et les interférences d'accouplement (les mâles {\em Ae. albopictus} recherchant plus agressivement l'accouplement avec des femelles {\em Ae. aegypti} que l'inverse, causant une baisse de la fécondité) \cite{juliano2005ecology}.
L'installation d'{\em Ae. albopictus} a ainsi été associée au déclin des populations d'{\em Ae. aegypti} en Amérique du Nord et au Brésil \cite{paupy2009aedes}.


\begin{figure}[t]
	\centering
	\includegraphics[width=15cm]{Figures/larval_breeding_sites_cDidier_Fontenille_IRD.pdf}
	\caption{Exemple de site de développement larvaire naturel (A) ou artificiel (B) d'{\em Aedes albopictus} (source : Didier Fontenille, IRD)}
	\label{fig:larval_breeding}
\end{figure}

\subsubsection{L'extension fulgurante d'{\em Ae. albopictus}}

Les avantages compétitifs d'{\em Ae. albopictus} permettent d'expliquer sa prééminence sur les espèces locales lorsqu'il est introduit dans un nouveau milieu présentant des conditions favorables.
De fait, de multiples introductions ont eu lieu dans le monde entier à partir des années 1960 avec la multiplication des échanges commerciaux.
En particulier, la dispersion d'\oe ufs ou de larves par le commerce international de pneus usagers semble avoir joué un rôle prépondérant \cite{reiter1998aedes}.
Originaire d'Asie, {\em Ae. albopictus} a ainsi été introduit et est devenu prédominant dans de très nombreuses zones du monde en quelques dizaines d'années seulement.
Il est aujourd'hui établit sur le continent Américain des États-Unis à l'Argentine, dans de nombreuses îles d'Océanie, en Afrique australe et centrale, en Europe méditerranéenne et poursuit sa progression vers le nord de la France, la Suisse, l'Allemagne et les Pays-Bas) (Fig. \ref{fig:repartitionaedesalbopictus}) \cite{kraemer2015global}.
Cette extension fulgurante peut être mise en parallèle avec celle, plus ancienne et plus lente d'{\em Ae. aegypti}, suivant les navigateurs de l'époque moderne.


\section[Émergences \& réémergences]{Émergences et réémergences}

La population humaine ainsi que les échanges internationaux ont connu une croissance sans précédent au cours du XX\textsuperscript{ème} siècle.
Alors que l'urbanisation est associée à l'augmentation des populations d'{\em Ae. aegypti}, l'augmentation des échanges a favorisé l'extension des zones d'activité d'{\em Ae. albopictus}. 
Ces phénomènes ont entraîné une multiplication des contacts entre vecteurs et populations humaines, qui s'est traduite par la résurgence d'épidémies de maladies anciennes comme la dengue et la fièvre jaune, mais aussi par la dissémination de virus initialement circonscrits à des zones géographiques limitées, comme les virus du chikungunya ou du Zika.
Dans cette partie, nous passerons en revue l'épidémiologie des principales maladies transmises par les moustiques du genre {\em Aedes}.


\subsection{Virus de la fièvre jaune}

La fièvre jaune est causée par un virus à ARN du genre {\em Flavivirus}.
Il s'agit d'une maladie très ancienne, originaire d'Afrique et introduite aux Amériques au XVI\textsuperscript{ème} ou au XVII\textsuperscript{ème} siècle, probablement durant la traite des esclaves \cite{bryant2007out}.
La maladie a été longtemps très redoutée à cause des épidémies dévastatrices qu'elle occasionnait dans les villes portuaires d'Afrique et des Amériques.
Par ce fait, cette maladie a contribué à façonner l'extension coloniale des pays européens à cette période.
La découverte du rôle d'{\em Ae. aegypti} dans la transmission de la fièvre jaune en 1900 a permis la mise en place de premières mesures de contrôle vectoriel, qui ont en particulier été appliquées lors du creusement du canal de Panama.
Le développement par Max Theiler d'un vaccin vivant atténué très efficace en 1937 (le vaccin 17D, efficace à 99\% et encore utilisé aujourd'hui) et sa large diffusion ont entraîné un franc déclin de la maladie, malgré la persistance de résurgences régulières dans certaines régions d'Afrique et d'Amérique centrale et du sud. 

En 2015, une épidémie importante a eu lieu à Luanda, en Angola puis s'est étendue au reste du pays et à la République Démocratique du Congo, causant plus de 7000 cas rapportés et près de 400 décès \cite{kraemer2017spread}. 
L'exportation de cas dans de nombreux pays du monde a fortement inquiété les autorités sanitaires et épuisé les stocks de vaccins \cite{barrett2016yellow}.
En 2016-2017, une épidémie d'ampleur inhabituelle a touché le Brésil, avec plus de 3500 cas rapportés et plusieurs centaines de décès \cite{moreira2018evidence}.
Les raisons de ces résurgences restent floues, les diminutions de couverture vaccinale ne permettant pas de les expliquer entièrement.
Plusieurs explications ont été avancées, dont l'augmentation de la densité et de l'activité des vecteurs, et des modifications génétiques du virus \cite{monath2015yellow}.
La maladie reste ainsi très présente dans les zones d'activité historiques d'Afrique et d'Amérique centrale et du sud (Fig. \ref{fig:yfworld}) \cite{shearer2018existing}.
De façon surprenante, aucune transmission autochtone de fièvre jaune n'a jamais pu être observée sur le continent asiatique, et ce malgré la présence du vecteur {\em Ae. aegypti}, et l'importation de nombreux cas infectieux lors de l'épidémie angolaise de 2015-2016 \cite{chen2016yellow}.

\begin{figure}[t]
	\centering
	\includegraphics[width=15cm]{Figures/yellow_fever_risk_.jpg}
	\caption{Distribution spatiale du risque individuel de fièvre jaune (source : Shearer et coll., 2018)}
	\label{fig:yfworld}
\end{figure}

Il existe deux formes de transmission, la première en lien avec un cycle sylvatique de la maladie comprenant des primates non-humains et des moustiques forestiers, atteignant sporadiquement des humains à proximité (appelée en anglais \guillemotleft jungle yellow fever\guillemotright); la seconde consiste en une transmission inter-humaine plus intense due à {\em Ae. aegypti} appelée \guillemotleft urban yellow fever\guillemotright \cite{monath2015yellow}.
Chez les humains, l'infection par le virus de la fièvre jaune se traduit par une fièvre intense accompagnée de nausées, de vomissements, de céphalées et myalgies, ce après une période d'incubation de 3 à 6 jours.
Ces symptômes durent 3 à 4 jours.
Dans environ 15\% des cas, une seconde phase toxique apparaît ensuite, caractérisée par un ictère lié à une insuffisance hépatique, des hémorragies oculaires et gastro-intestinales, parfois compliquées d'insuffisance rénale et de syndrome de choc.
Dans ce cas, le décès dans 20 à 60\% des cas \cite{monath2015yellow}.
La mortalité est plus faible en Afrique qu'en Amérique du sud, ce qui pourrait s'expliquer par des adaptations génétiques (\guillemotleft co-évolution\guillemotright) à cette maladie originaire du continent Africain.
L'infection confère une immunité à vie.



\subsection{Virus de la dengue}
\label{sec:dengue}

Le virus de la dengue (généralement abrégé en DENV) est un virus à ARN appartenant au genre {\em Flavivirus} et qui comprend cinq sérotypes différents (le dernier, DENV-5, ayant été identifié en 2013).
Il semble que les différents sérotypes de dengue aient divergé il y a environ mille ans, mais n'aient établi de transmission endémique chez l'humain que depuis quelques centaines d'années \cite{holmes2003origin}.
Auparavant, une forme sylvatique du virus circulait dans les populations de primates non-humains d'Afrique et d'Asie.
Les premières épidémies de maladies cliniquement semblables à la dengue ont été décrites vers la fin du XVIII\textsuperscript{ème} siècle en Asie et aux Amériques.
A la fin du XIX\textsuperscript{ème} siècle, la dengue était présente dans de nombreuses zones tropicales et subtropicales, et le virus a pu être isolé pour la première fois au Japon en 1943 (DENV-1) et à Hawaï en 1945 (DENV-2).
Par la suite, la maladie est devenue endémique dans les pays d'Asie du sud et du sud-est et d'Amérique centrale du sud (Fig. \ref{fig:denvworld}), avec une augmentation importante du nombre global de cas rapportés à partir des années 1990 pour atteindre 60 millions d'infections symptomatiques et 10 000 décès par an en 2013 (Fig. \ref{fig:denvaug})  \cite{messina2014global,stanaway2016global,bhatt2013global}.

\begin{figure}[t]
	\centering
	\includegraphics[width=15cm]{Figures/DENVworld_bhatt2.jpg}
	\caption{Cartogramme du nombre de cas de dengue dans le monde (source : Bhatt et coll., 2013)}
	\label{fig:denvworld}
\end{figure}


\begin{figure}[t]
	\centering
	\includegraphics[width=15cm]{Figures/dali_gbd_dengue.png}
	\caption{\'Evolution du fardeau de la dengue (mesuré en espérance de vie corrigée de l'incapacité) entre 1990 et 2013 (source : Stanaway et coll., 2016)}
	\label{fig:denvaug}
\end{figure}

L'infection par DENV est asymptomatique dans 75 à 90\% des cas, mais peut aussi se traduire après une période d'incubation de trois à sept jours par une période symptomatique en trois phases \cite{simmons2012dengue}.
Une première phase est dominée par une fièvre indifférenciée $\geq$38.5$^{\circ}$C, parfois accompagnée de myalgies, d'arthralgies et d'éruptions cutanées.
Ces symptômes durent pendant trois à sept jours.
Vers la fin de cette période, l'infection peut dans certains cas causer un syndrome de fièvre hémorragique, caractérisé par une hémoconcentration avec thrombocytopénie, une hypoprotéinémie, des épanchements pleuraux et péritonéaux, et parfois des manifestations hémorragiques.
L'hypotension signe la complication en syndrome de choc, qui doit être recherché activement, et constitue le principal danger de la maladie, qui est fatale dans moins de 1\% des cas.
Ces troubles, liés à une altération de la perméabilité vasculaire, sont généralement spontanément résolutifs en deux ou trois jours sous traitement symptomatique.
Un asthénie peut persister pendant plusieurs semaines.

L'infection par un sérotype donné confère une immunité temporaire contre tous les sérotypes, et une immunité à long terme contre ce sérotype seulement.
La survenue de cas graves est plus fréquente si le sujet a été infecté par un autre sérotype dans le passé, en particulier chez les enfants, un phénomène appelé \textit{antibody-dependent enhancement} \cite{halstead1970observations}.
Cette caractéristique a fortement compliqué la mise au point d'un vaccin, par crainte qu'il ne favorise l'apparition de formes graves.
Un premier vaccin nommé Dengvaxia a été mis sur le marché en 2016.
Une campagne de vaccination de masse dans les écoles, lancée aux Philippines, a toutefois du être suspendue fin 2017 après des suspicions de décès en lien avec des formes sévères de la dengue après vaccination \cite{denguevacc}.



\subsection{Virus du chikungunya}

Contrairement aux virus de la dengue et de la fièvre jaune, le virus du chikungunya ne fait pas partie du groupe des {\em Flavivirus} mais de celui des {\em Alphavirus}.
Son nom provient de la langue Makonde (parlée en Tanzanie et au Mozambique) et signifie \guillemotleft maladie de l'homme courbé\guillemotright, en lien avec l'arthralgie caractéristique de la maladie.
Il a été isolé pour la première fois durant une épidémie en Tanzanie en 1953 \cite{ross1956newala}.
Le virus est originaire des forêts d'Afrique subsaharienne, où il suit un cycle ancestral de type enzootique, incluant des primates non-humains et des moustiques sylvestres \cite{weaver2015chikungunya}.
On distingue deux types de souches enzootiques, une souche ouest-africaine et une souche présente dans le reste de l'Afrique subsaharienne nommée {\em ECSA}  (pour \guillemotleft Eastern, Central and Southern Africa\guillemotright).
Des études phylogénétiques suggèrent l'existence de multiples transitions entre l'état enzootique et un état épidémique incluant une transmission interhumaine plus soutenue par {\em Aedes aegypti}.
On retrouve ainsi les traces de nombreuses épidémies dès le XVII\textsuperscript{ème} siècle, au Caire et aux Indes Néerlandaises (1779), à Zanzibar (1823 et 1870), en Inde (1823, 1824–1825 et 1871–1872), à Hong Kong et à Madras (1901–1902) \cite{carey1971chikungunya}.

Une transition ce type a donné naissance à une souche asiatique, depuis une souche de type ECSA, probablement au début du XX\textsuperscript{ème} siècle \cite{volk2010genome}.
Cette souche asiatique a été à l'origine d'épidémies régulières en Inde et en Asie du sud-est dans les années 1960 et 1970.
Un deuxième évènement similaire s'est produit en 2004 au Kenya, donnant naissance à la souche dite \textit{IOL} (pour \guillemotleft Indian Ocean Lineage\guillemotright) \cite{chretien2007drought}. 
Le virus s'est propagé à plusieurs îles de l'océan Indien, dont la Réunion où l'épidémie été particulièrement bien observée, grâce à un système de surveillance efficace.
Au total, entre janvier 2005 et avril 2006, le nombre d'infections a été estimé à 244 000 cas, en deux vagues d'ampleurs très différentes (Fig. \ref{fig:chiklareuion}) \cite{renault2007major}.
La même souche est ensuite retrouvée en Inde et en Asie du sud-est, menant à des épidémies massives causant plusieurs millions de cas au total \cite{arankalle2007genetic}.
Ces épidémies ont occasionné de nombreux cas d'importation dans le monde entier, qui ont en particulier donné lieu à des épisodes de transmission autochtone du virus en Europe, dans des zones où la présence d'{\em Ae. albopictus} est connue.
Un premier épisode de ce type est survenu en 2007 dans deux villages de la région d'\'Emilie-Romagne, en Italie, avec 161 cas confirmés et un taux d'attaque estimé à 10,2\% \cite{poletti2011transmission}.
Un deuxième épisode a eu lieu en 2010 à Fréjus, en France, avec deux cas autochtones confirmés \cite{grandadam2011chikungunya}.
Une adaptation génétique de la souche IOL vers une plus grande infectivité de {\em Ae. albopictus} peut expliquer en partie l'explosivité de ces épidémies, survenant subitement alors que la souche asiatique circulait dans ces zones depuis au moins plusieurs dizaines d'années \cite{arankalle2007genetic,tsetsarkin2007single,tsetsarkin2011chikungunya}.
\begin{figure}[t]
	\centering
	\includegraphics[width=11cm]{Figures/chik_lareunion.png}
	\caption{Courbe épidémique en deux vagues successives du chikungunya à la Réunion (source : Renault et coll., 2007)}
	\label{fig:chiklareuion}
\end{figure}

En 2013, la souche asiatique du virus a été introduite sur l'île de Saint-Martin, aux Antilles françaises, et s'est propagée à toutes les zones tropicales et semi-tropicales du continent américain, puis vers les îles de l'océan Pacifique \cite{leparc2014chikungunya}.
Parallèlement en 2014, un virus appartenant à la souche ECSA a été isolée au Brésil \cite{nunes2015emergence}.
Entre 2013 et 2016, 1,9 millions le nombre de cas suspects ont été rapportés à l'Organisation Panaméricaine de la Santé (PAHO).
L'histoire de l'extension du virus est résumée dans la figure \ref{fig:chikworld}.

\begin{figure}[t]
	\centering
	\includegraphics[width=15cm]{Figures/weaver_chik.png}
	\caption{Origine, extension et distribution du virus du chikungunya et de ses vecteurs (source : Weaver et coll., 2015)}
	\label{fig:chikworld}
\end{figure}

Les symptômes de l'infection par le virus du chikungunya comprennent une fièvre élevée, une asthénie, des arthralgies et myalgies, et une éruption cutanée \cite{weaver2015chikungunya}. 
Ces signes sont présents chez environ 85\% des sujets infectés, surviennent après une période d'incubation de 3 jours et durent entre 7 et 10 jours.
Les complications graves sont rares, et concernent le plus souvent des sujets à risque comme les patients âgés, nouveaux-nés, ou porteurs d'autres pathologies.
L'asthénie et les douleurs articulaires persistent parfois plusieurs mois chez près de 60\% des sujets atteints \cite{schilte2013chikungunya}.
L'infection confère une immunité à long terme. 


\subsection{Virus Zika}

Le virus Zika appartient au genre {\em Flavivirus}, comme ceux de la dengue et de la fièvre jaune.
Il a été isolé pour la première fois en 1947 chez un singe rhésus sentinelle placé dans la forêt Zika, en Ouganda, pour la surveillance virologique de la fièvre jaune \cite{dick1952zika}.
L'année suivante, le virus est aussi isolé chez un moustique {\em Aedes africanus} de la même forêt.
En 1954, le premier cas humain est rapporté au Nigéria, mais des études ultérieures montrent qu'il s'agit en fait d'un autre virus proche, nommé Spondweni \cite{wikan2016zika}.
Le premier cas humain est ainsi rapporté en 1964 par Simpson, qui décrit sa propre infection \cite{simpson1964zika}.
Des études ont par la suite montré la présence du virus en Afrique subsaharienne, en Egypte, en Inde et en Asie du sud-est (Fig. \ref{fig:zika2007}).
Toutefois, seulement 13 cas humains ont pu être identifiés avant l'année 2007, probablement en lien avec une circulation sylvatique impliquant des primates non-humains et occasionnant sporadiquement des cas humains à proximité.
En 2007 survient la première épidémie importante de Zika sur l'île de Yap (Micronésie) avec 108 cas confirmés ou suspects (Fig. \ref{fig:yap2007}) \cite{duffy_zika_2009}.
\begin{figure}[t]
	\centering
	\includegraphics[width=9cm]{Figures/yap.png}
	\caption{Courbe épidémique du Zika sur l'île de Yap en 2007 (source : Duffy et coll., 2009)}
	\label{fig:yap2007}
\end{figure}
De 2010 à 2013, plusieurs cas isolés sont identifiés en Asie du sud-est, ou chez des voyageurs revenant de cette région, possiblement en lien avec des épidémies à bas bruit \cite{wikan2016zika}.
A partir d'octobre 2013, des épidémies sont rapportées dans plusieurs îles de Polynésie française, puis à Vanuatu, aux îles Cook, en Nouvelle-Calédonie et à l'île de Pâcques \cite{cao2014zika,wikan2016zika}.
Les analyses virologiques montrent une proximité génétique entre les virus circulant en Polynésie et ceux circulant à Yap et en Asie du sud-est.
Puis, en mars 2015, des cas de Zika ont été rapportés dans les états de Bahia et de Rio Grande do Norte (Brésil), montrant là aussi une proximité avec les virus circulant en Polynésie.
L'introduction du virus dans ce pays a pu être reliée à l'organisation de la coupe du monde de football en juin 2014 ou d'une compétition internationale de canoë en août 2014 \cite{musso_zika_2015}.
Du Brésil, l'épidémie s'est propagée à toute l'Amérique du sud et à l'Amérique centrale, occasionnant plusieurs millions de cas, ce qui a conduit l'OMS a déclarer un état d'urgence sanitaire en février 2016.
L'adaptation génétique du virus Zika vers une plus grande infectivité de {\em Ae. aegypti} a pu jouer un rôle dans cette extension \cite{liu2017evolutionary}.

\begin{figure}[t]
	\centering
	\includegraphics[width=15cm]{Figures/zikaspread.png}
	\caption{Propagation mondiale du virus Zika déduite à partir d'analyses phylogénétiques ou de cas d'importations identifiés (source : Gatherer et Kohl, 2016)}
	\label{fig:zika2007}
\end{figure}

Le principal mode d'infection par le virus Zika est la piqûre par un moustique infecté du genre {\em Aedes}, dont {\em Ae. aegypti} et {\em Ae. albopictus} \cite{cornet1979transmission,wong2013aedes}.
La transmission est aussi possible par voie sexuelle, comme l'ont montré plusieurs cas survenus chez des conjoints de cas importés en Europe et aux \'Etats-Unis d'Amérique.
Toutefois, même si l'incertitude demeure, il semble que cette voie de transmission soit peu efficace et ne contribue que marginalement aux épidémies de Zika \cite{althaus_how_2016}.
L'infection par le virus cause des symptômes dans environ 20\% des cas, après une période d'incubation d'environ 6 jours \cite{lessler2016times}.
L'infection confère une immunité à long terme. 
Les symptômes se limitent généralement à une fièvre avec éruption maculo-papuleuse et arthralgies, parfois accompagnées de myalgies, de céphalées et de conjonctivite, et durent une à deux semaines \cite{lessler2016assessing}.
Les décès sont très rares, mais peuvent survenir chez des sujets atteints de comorbidités ou d'immunodépression.
Des complications neurologiques peuvent subvenir, dont un risque de syndrome de Guillain-Barré estimé en Polynésie française à 24 pour 100 000 infections \cite{cao2016guillain}, mais aussi un lien suspecté avec la méningo-encéphalite et avec la myélite transverse \cite{lessler2016assessing}.
Toutefois, l'aspect le plus inquiétant des épidémies massives de Zika réside dans le lien entre l'infection de femmes enceintes et la survenue de malformations congénitales chez l'enfant à naître, en particulier la microcéphalie, mais aussi des calcifications intracrâniennes, des anomalies oculaires, ou encore une ventriculomégalie.
L'analyse de cas de microcéphalie identifiés en Polynésie française suggère un risque d'environ 1\% pour les femmes enceintes infectées durant le premier semestre de grossesse \cite{cauchemez_association_2016}.

\begin{figure}[t]
	\centering
	\includegraphics[width=15cm]{Figures/lessler_cycle.png}
	\caption[Exemples de différents cycles locaux de transmission du virus Zika (source : Lessler et coll., 2016)]{Exemples de différents cycles locaux de transmission du virus Zika: (A) en Suède, la transmission vectorielle du virus n'est pas possible, et le seul risque est lié à l'importation de cas infectés (avec une possible transmission secondaire limitée par voie sexuelle); (B) au Brésil, où les vecteurs capables de transmettre efficacement le virus d'humain à humain ({\em Ae. aegypti}) sont abondants, l'importation de cas infectés peut entraîner des épidémies importantes; (C) au Nigéria, des cas humains sont régulièrement identifiés, sans qu'on sache avec certitude si une transmission inter-humaine existe ou s'il s'agit de cas répétés de débordement depuis un cycle enzootique (\textit{effet spillover}); (D) au Sénégal, le virus semble se maintenir dans un cycle enzootique sylvatique incluant des moustiques forestiers et des primates non-humains (source : Lessler et coll., 2016).}
	\label{fig:lesslercycle}
\end{figure}

\section[Virus émergents du futur]{Les virus émergents du futur}
\label{sec:emerg}
Ces émergences de maladies transmises par les moustiques du genre {\em Aedes} sont les conséquences de processus complexes, où toutefois un schéma général se dessine.
On retrouve un virus circulant originellement dans un cycle enzootique, souvent en milieu forestier, impliquant des moustiques spécifiques et un réservoir animal (des primates non-humains pour chacun des virus décrits précédemment).
Dans un tel cycle, des cas humains surviennent sporadiquement dans les communautés humaines vivant à proximité à l'occasion de piqûres par ces moustiques forestiers, dans un phénomène appelé \textit{effet spillover}.
Toutefois, ces infections sont généralement sans suite, les moustiques concernés ne piquant les humains que de façon occasionnelle, et les moustiques plus spécifiques des humains n'ayant pas de compétence vectorielle pour ces virus.
Ce type de schéma, existant depuis des temps reculés pour la fièvre jaune en Afrique par exemple, mène à l'observation de cas sporadiques dans des populations rurales sur de longues durées (Fig. \ref{fig:zika2007}C et D).
La situation est modifiée par la présence de fortes densités de moustiques adaptés à la vie à proximité des habitats humains, en particulier {\em Ae. aegypti} et {\em Ae. albopictus}.
La multiplication des contacts entre ces vecteurs et des humains transitoirement infectés par des virus crée une pression de sélection, favorisant l'adaptation génétique des virus \guillemotleft sauvages\guillemotright\ à la transmission par les moustiques \guillemotleft domestiques\guillemotright, et \textit{in fine} à l'apparition d'une transmission inter-humaine efficace et donc à des épidémies (Fig. \ref{fig:zika2007}B).
L'extension géographique de la maladie nouvelle est alors fonction des échanges internationaux et de la présence ou non du vecteur compétent.
Cette suite d'événements semble s'être produite anciennement pour la fièvre jaune et la dengue, beaucoup plus récemment et rapidement pour le Zika et le chikungunya.
Elle a pu être mise en évidence plus formellement pour la souche IOL de chikungunya \cite{tsetsarkin2011chikungunya} et pour le Zika \cite{liu2017evolutionary}.
\begin{table}[p]
	\centering
	\caption{Principaux virus à risque d'émergence future. \vspace{.5em}}
	\label{table:candidats}
	\begin{tabular}{L{2.3cm}L{1.1cm}L{2cm}L{2.2cm}L{2.3cm}L{3.5cm}}
		\hline 
Virus & Genre &	Zone  & Réservoir & Vecteur principal & Commentaires \\
\hline
Mayaro & { A$^{*}$ } & Amérique du sud, Amérique centrale  & Primates non-humains & {\em Haemagogus spp.} & possible transmission inter-humaine par {\em Ae. aegypti} \cite{long2011experimental,lednicky2016mayaro} \\[.8em]

Usutu & { F$^{\dagger}$} & Afrique, Europe & Oiseaux & {\em Culex pipiens}& circulation parmi les humains en Europe, isolé chez {\em Ae. albopictus} \cite{percivalle2017usutu,nikolay2015review} \\[.8em]

Ross-River & { A} & Océanie & Kangourous, autres mammifères & {\em Culex spp.},  {\em Aedes spp.} & compétence vectorielle expérimentale d'{\em Ae. aegypti} et d'{\em Ae. albopictus} \cite{nasci1994larval,mitchell1987vector,lau2017new} \\[.8em]

Encéphalite équine de l'est & { A} & Amérique du nord & Oiseaux & {\em Culiseta melanura} & épizooties régulières chez les chevaux, isolé chez {\em Ae. albopictus} \cite{molaei2006identification,mitchell1992isolation}\\[.8em]

Nil occidental & { F} & Afrique, Europe, Asie, Océanie, Amérique du nord & Oiseaux & {\em Culex spp.} & épidémies en Amérique du nord (1999-2000), isolé chez {\em Ae. aegypti} \cite{campbell2002west,turell2001potential,kramer2007west} \\[.8em]

Encéphalite japonaise & { F} & Asie, Océanie & Cochons, oiseaux & {\em Culex tritaeniorhynchus} & 30 à 50 000 cas annuels identifiés, isolé chez {\em Ae. albopictus} \cite{mackenzie2004emerging,buescher1959ecologic,turell2001potential}\\[.8em]

Encéphalite de Saint Louis & { F} & Amérique du nord & Oiseaux & {\em Culex spp.} & isolé chez {\em Ae. albopictus} \cite{cdc2018saintlouis}\\[.8em]

Encéphalite de La Crosse & { B$^{\ddagger}$} & Amérique du nord & Écureuils &  {\em Ae. triseriatus} & isolé chez {\em Ae. albopictus} \cite{cdc2018lacrosse,gerhardt2001first}\\[.8em]

\hline
\multicolumn{6}{l}{\footnotesize{\textit{$^{*}$Alphavirus; $^{\dagger}$Flavivirus; $^{\ddagger}$Bunyavirus}}}

	\end{tabular} 
\end{table}


La situation actuelle de répartition et d'abondance des moustiques {\em Ae. aegypti} et {\em Ae. albopictus} rend probable que d'autres virus suivent la même voie.
L'émergence de virus nouveaux dans des populations entièrement susceptibles pourrait causer des épidémies massives, avec des conséquences difficile à prévoir.
Cela constitue un véritable problème de santé globale, qui concerne déjà la plupart des zones tropicales du monde, et qui pourrait, compte-tenu du changement climatique, toucher à son tour des régions plus tempérées. 
On s'attend en effet à une extension de la zone d'activité d'{\em Ae. albopictus} et d'{\em Ae. aegypti}, en particulier en Europe centrale et occidentale et en Amérique du nord \cite{fischer2011projection,rochlin2013climate,monaghan2018potential}.
Dans ce contexte, il est important d'améliorer l'état des connaissances d'une part sur les maladies susceptibles d'émergence future, et d'autre part sur les caractéristiques générales des dynamiques épidémiques de ces infections, afin d'améliorer la préparation des acteurs de santé publique.
Nous proposons dans le tableau \ref{table:candidats} un bref inventaire des principaux virus connus présentant ces caractéristiques.


%
%\subsubsection*{Virus de l'encéphalite La Crosse}
%
%Le virus La Crosse, du genre {\em Bunyavirus}, a été isolé pour la première fois en 1963 dans le cerveau d'un enfant de 4 ans décédé d'encéphalite dans la ville de La Crosse, Wisconsin, et est présent dans la moitié est des \'Etats-Unis d'Amérique \cite{mcjunkin2001crosse}.
%En moyenne, 63 cas sont reportés chaque année dans ce pays \cite{cdc2018lacrosse}.
%Le principal vecteur est {\em Aedes triseriatus}, mais il a été aussi retrouvé chez {\em Aedes albopictus} \cite{gerhardt2001first}.
%Les symptômes surviennent après 5 à 15 jours, et comprennent fièvre, céphalées, nausée, et léthargie \cite{cdc2018lacrosse}.
%L'atteinte neurologique sévère est plus courante chez les enfants, mais est rarement fatale ($<$1\%).
%

%
%\subsubsection*{Virus Mayaro}
%
%Le virus Mayaro appartient au genre {\em Alphavirus}, comme le chikungunya, et circule de façon enzootique dans les forêts d'Amérique du sud, en particulier en Amazonie. 
%Son vecteur principal appartient au genre {\em Haemagogus}, mais il a été démontré expérimentalement qu'il est capable d'être transmis par {\em Ae. aegypti} \cite{long2011experimental}.
%Jusqu'à récemment, les seuls cas humains avaient été rapportés dans des communautés vivant à proximité de la forêt amazonienne, ou chez des touristes ayant réalisé des séjours en forêt \cite{hotez2017dengue}.
%En 2016, le virus a été retrouvé chez un garçon vivant dans une zone non-forestière d'Haïti, qui présentait une co-infection avec le virus de la dengue, suggérant une transmission inter-humaine par {\em Ae. aegypti} \cite{lednicky2016mayaro}.
%Chez l'humain, l'infection provoque des symptômes similaires à ceux du chikungunya, incluant fièvre, éruption et arthralgies sévères.
%
%
%\subsubsection*{Virus Usutu}
%
%Le virus Usutu appartient au genre {\em Flavivirus}, et a été découvert en Afrique du sud en 1959, puis dans plusieurs pays d'Afrique de l'est, où deux cas humains ont pu être identifiés \cite{nikolay2011usutu}.
%Le virus a ensuite été retrouvé en Autriche en 2001, où il a causé des épizooties chez plusieurs espèces d'oiseaux (en particulier des merles), puis dans d'autres pays d'Europe.
%Deux cas humains d'infection, présentant un tableau d'encéphalite sévère, ont été identifiés en Italie en 2009 puis en France en 2016 \cite{pecorari2009first,simonin2018human}.
%Des études de séroprévalence ont aussi montré la circulation du virus dans les populations humaines d'Europe \cite{percivalle2017usutu}.
%Le virus a aussi été isolé chez plusieurs espèces de moustiques, dont {\em Ae. albopictus} \cite{nikolay2015review}.

%
%\subsubsection*{Virus Ross-River}
%
%Le virus Ross-River est un virus du genre {\em Alphavirus}, endémique en Australie et en Papouasie-Nouvelle-Guinée.
%De nos jours, près de 5000 cas humains sont rapportés chaque année dans ces zones \cite{lau2017new}.
%Les symptômes sont présents dans 25 à 50\% des cas, et comprennent fièvre, éruption et arthralgies sévères pouvant persister plusieurs mois.
%Le virus suit un cycle naturel où les principaux réservoirs sont les kangourous et les principaux vecteurs des moustiques des genres  {\em Culex} et {\em Aedes}. 
%Une épidémie massive a touché l'ensemble des pays de la région du Pacifique sud en 1979, avec plus de 500 000 cas rapportés \cite{aaskov1981epidemic}. 
%Cet événement démontre la possibilité de transmission inter-humaine, court-circuitant les réservoirs naturels.
%La capacité vectorielle d'{\em Ae. aegypti} et d'{\em Ae. albopictus} a pu être montrée expérimentalement \cite{nasci1994larval,mitchell1987vector}.
%Des étude de séroprévalence ont aussi montré la présence du virus en Polynésie française et aux îles Samoa, où les marsupiaux sont absents, ce qui suggère l'existence d'autres réservoirs animaux (chiens, rats, cochons, chats ou chauve-souris) et donc la possibilité de l'extension de la zone de circulation du virus \cite{lau2017new}.

%\subsubsection*{Virus de l'encéphalite équine de l'est }
%
%Le virus de l'encéphalite équine de l'est appartient au genre {\em Alphavirus}.
%Il circule naturellement dans les zones marécageuses de l'est des \'Etats-Unis d'Amérique dans un cycle incluant des oiseaux sauvages et le moustique {\em Culiseta melanura}.
%La transmission aux chevaux ou aux humains est due à la piqûre inhabituelle de mammifères par ce moustique ou par l'intervention d'autres vecteurs, incluant certains moustiques du genre {\em Aedes}, en particulier {\em Ae. vexans} et {\em Ae. albopictus} \cite{molaei2006identification,mitchell1992isolation}.
%Le virus est à l'origine d'épizooties régulières chez les chevaux, et d'infections humaines sporadiques.
%Ni la transmission inter-humaine, ni la transmission de chevaux à humains n'ont pu être démontrées.
%Chez l'humain, les symptômes surviennent 4 à 10 jours après l'infection dans une fraction des sujets, et comprennent une fièvre, des céphalées, des nausées et vomissements, des myalgies et des signes neurologiques de méningo-encéphalite \cite{deresiewicz1997clinical}.
%La mortalité est de 36\% en 2 à 10 jours après l'apparition des symptômes, et des séquelles subsistent chez 35\% des survivants \cite{deresiewicz1997clinical}.
%
%\subsubsection*{Virus du Nil occidental}
%
%Le virus du Nil occidental est un {\em Flavivirus}, isolé pour la première fois en 1937 chez un habitant de la province du Nil occidental, en Ouganda.
%A l'état naturel, le virus se maintient dans un cycle enzootique comprenant des oiseaux et des moustiques du genre {\em Culex}.
%Toutefois, le virus a été isolé chez de nombreuses espèces de moustiques, dont {\em Ae. albopictus} \cite{turell2001potential}.
%Il est présent en Afrique, en Europe, en Asie, en Australie et a été introduit en Amérique du nord à la fin du XX\textsuperscript{ème} siècle \cite{campbell2002west}.
%Ce virus a causé plusieurs épidémies incluant une transmission inter-humaine,notamment en Afrique du sud en 1973-1974, à Bucarest en Roumanie en 1996, et dans la ville de New York aux \'Etats-Unis d'Amérique en 1999-2000.
%L'infection d'un humain est symptomatique dans 20 à 40\% des cas, et dans ce cas se traduit par une fièvre avec céphalées, asthénie, arthralgies et éruption cutanée, après une période d'incubation de 2 à 14 jours \cite{kramer2007west}.
%L'atteinte neurologique survient dans moins de 1\% des infections, et se caractérise par un tableau d'encéphalite, de méningo-encéphalite ou de paralysie aiguë.
%Dans ce cas, le taux de décès est inférieur à 10\% \cite{kramer2007west}.
%
%
%\subsubsection*{Virus de l'encéphalite japonaise}
%
%Le virus de l'encéphalite japonaise est un {\em Flavivirus}, présent dans les zones rurales de l'est, du sud et du sud-est de l'Asie, ainsi qu'en Papouasie-Nouvelle-Guinée et au nord de l'Australie.
%C'est la cause la plus importante d'encéphalite dans ces régions, avec 30 à 50 000 cas annuels identifiés \cite{mackenzie2004emerging}.
%Le virus existe dans un cycle enzoonotique impliquant des moustiques et cochons ou des oiseaux aquatiques \cite{buescher1959ecologic}.
%Le vecteur le plus important est {\em Culex tritaeniorhynchus}, mais le virus a été isolé chez {\em Ae. albopictus} et {\em Ae. aegypti}.
%La proportion d'infection symptomatiques est de moins de 0.1 à 4\%.
%La période d'incubation est de 5 à 15 jours, et les symptômes comprennent une fièvre aspécifique, avec des symptômes neurologiques souvent sévères, en particulier un syndrome Parkinsonien et des paralysies périphériques \cite{mackenzie2004emerging}.
%Parmi les cas symptomatiques, le taux de décès est de 30\% et le taux de séquelles neurologiques permanentes de 50\%. 
%La maladie peut être prévenue par un vaccin, qui est recommandé chez les personnes voyageant dans les zones endémiques.
%
%
%\subsubsection*{Virus de l'encéphalite Saint-Louis}
%
%Le virus de l'encéphalite Saint-Louis, membre du genre {\em Flavivirus}, est fréquent dans l'ensemble du continent américain.
%Le virus existe dans un cycle enzootique incluant des oiseaux et des moustiques du genre {\em Culex}, et a été retrouvé chez {\em Aedes albopictus}.
%
%La plupart des infections sont asymptomatiques. 

%\subsubsection*{Virus Jamestown Canyon}
%
%\subsubsection*{Virus de l'encéphalite de Murray Valley}
%
%%
%%\subsubsection*{Virus de l'encéphalite La Crosse}
%%
%%Le virus La Crosse, du genre {\em Bunyavirus}, a été isolé pour la première fois en 1963 dans le cerveau d'un enfant de 4 ans décédé d'encéphalite dans la ville de La Crosse, Wisconsin, et est présent dans la moitié est des \'Etats-Unis d'Amérique \cite{mcjunkin2001crosse}.
%%En moyenne, 63 cas sont reportés chaque année dans ce pays \cite{cdc2018lacrosse}.
%%Le principal vecteur est {\em Aedes triseriatus}, mais il a été aussi retrouvé chez {\em Aedes albopictus} \cite{gerhardt2001first}.
%%Les symptômes surviennent après 5 à 15 jours, et comprennent fièvre, céphalées, nausée, et léthargie \cite{cdc2018lacrosse}.
%%L'atteinte neurologique sévère est plus courante chez les enfants, mais est rarement fatale ($<$1\%).
%%

\section[Prévention \& contrôle]{Prévention et contrôle}

L'importance pour la santé globale des maladies transmises par les moustiques du genre {\em Aedes} appelle à une réponse coordonnée des acteurs de santé publique du monde entier.
Toutefois, l'efficacité des mesures de prévention et de contrôle est limitée \cite{lindsay2017improving}.
Le traitement des personnes infectées se limite à une prise en charge symptomatique, aucun traitement spécifique n'étant disponible.
Un vaccin préventif existe contre la fièvre jaune, mais les stocks mondiaux ont été affectés par les récentes épidémies d'Angola et du Brésil \cite{barrett2016yellow}.
Quant au vaccin récemment introduit contre la dengue, il a récemment subit des déboires \cite{denguevacc}.
Historiquement, le principal moyen de contrôle a été la lutte antivectorielle, basée sur la surveillance des sites larvaires et leur inactivation l'utilisation d'huile et la pulvérisation d'insecticides très puissants (en particulier le DDT, interdit dans de nombreux pays à partir des années 1970 pour des raisons environnementales).
Ces actions ont notamment permis l'élimination pratiquement totale de la fièvre jaune en Amérique du sud dans les années 1960, ainsi que de la dengue en Asie du sud-est et aux Caraïbes dans les années 1970 et 1980 \cite{camargo1967history}.
Malheureusement, ces mesures se sont révélées insuffisantes à contenir la résurgence des maladies transmises par les moustiques du genre {\em Aedes} à partir des années 1990 \cite{gubler2012dengue}.
Cet échec est lié à la diminution des efforts de contrôle, alors même que l'urbanisation croissante et l'augmentation des échanges internationaux favorisait l'extension et la prolifération des moustiques et qu'apparaissait le problème de la résistance aux insecticides \cite{vontas2012insecticide}.

De nos jours, il existe une grande diversité dans les mesures antivectorielles en application suivant les régions, sans que les actions soient toujours concertées ou basées sur des preuves suffisantes (Fig. \ref{fig:achee}) \cite{achee2015critical}.
On peut distinguer les mesures visant les moustiques au stade aquatique :
\begin{itemize}
\item éviction ou nettoyage des conteneurs artificiels susceptible de servir à la reproduction des moustiques (gestion des déchets, nettoyage à l'eau de javel, mise en place de couvercles) ;
\item surveillance et traitement spécifiques dans les sites larvaires irréductibles (insecticides spécifiques et écologiquement viables, agents biologiques se nourrissant de larves) ;
\item campagnes d'information dans les populations exposées ;
\item aspects législatifs : gestion de l'environnement urbain, normes de construction ;
\end{itemize}
de celles visant les moustiques adultes :
\begin{itemize}
\item pulvérisations aériennes d'insecticides visant à éliminer les moustiques adultes ;
\item pulvérisation des surfaces de repos des moustiques à l'intérieur des habitations ;
\item utilisation de moustiquaires ou de répulsifs (DEET) visant à réduire les piqûres.
\end{itemize}
De nouvelles méthodes sont aussi en développement, comme par exemple la relâche de mâles génétiquement modifiés pour transmettre un gène létal à leur descendance (RIDL pour {\em Release of Insects carrying a Dominant Lethal}), l'utilisation d'agents biologiques infectant les moustiques du genre {\em Aedes} (en particulier les bactéries du genre {\em Wolbachia}), la dispersion de leurres sucrés empoisonnés et la création d'insecticides plus spécifiques et moins susceptibles de favoriser la survenue de résistances.

\begin{figure}[t]
	\centering
	\includegraphics[width=16cm]{Figures/achee_vector_control.PNG}
	\caption{Méthodes de lutte antivectorielles visant {\em Ae. aegypti} et {\em Ae. albopictus} existantes (zone verte) ou méthodes en développement (zone jaune) selon leur mode d'action (source : Achee et coll., 2015).}
	\label{fig:achee}
\end{figure}

L'Organisation Mondiale de la Santé a dès 2012 proposé une stratégie globale de prévention et de contrôle de la dengue sur la période 2012-2020 \cite{who2012global}, élargie à toutes les maladies transmises par des vecteurs en 2017 \cite{who2017global}.
Les objectifs visés sont la réduction de la mortalité globale de 75\% et la réduction de la morbidité globale de 60\% d'ici à 2030.
La stratégie s'articule autour de cinq éléments:
\begin{itemize}
\item améliorer le diagnostic et la prise en charge des cas, avec pour objectif de réduire les conséquences de l'infection, en particulier la mortalité de la dengue;
\item généraliser les systèmes de surveillance épidémiologique nationaux afin d'améliorer la préparation des acteurs de santé publique, de permettre des interventions précoces en cas d'épidémie et d'évaluer l'efficacité des programmes;
\item instaurer sur le long terme des mesures de lutte antivectorielle adaptées et validées visant {\em Ae. aegypti} et {\em Ae. albopictus}, afin de diminuer le nombre et l'activité des vecteurs et de contenir la survenue de résistance aux insecticides;
\item augmenter la couverture vaccinale pour la dengue et la fièvre jaune;
\item donner plus de moyens à la recherche fondamentale et opérationnelle.
\end{itemize}



