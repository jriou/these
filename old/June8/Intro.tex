\chapter*{Introduction}
\chaptermark{Introduction}
\addcontentsline{toc}{chapter}{Introduction}

Le genre {\em Aedes} regroupe de nombreuses espèces de moustiques, vecteurs de nombreuses maladies humaines et animales. 
Parmi celles-ci, {\em Ae. aegypti} et {\em Ae. albopictus} ont une influence particulièrement importante sur la santé publique globale, du fait de leur ubiquité dans les zones tropicales et semi-tempérées.
L'abondance de ces moustiques à proximité des centres de population humaine peut être reliée à la résurgence des maladies transmises par ces vecteurs depuis le début du XXI\textsuperscript{ème} siècle.
Dans un premier chapitre, nous reviendrons sur les phénomènes historiques et entomologiques ayant mené à la situation actuelle de densité et de distribution géographique de ces moustiques.
Nous y listerons aussi les maladies transmises par ces vecteurs causant le plus grand fardeau pour la santé globale, dont les émergences récentes des virus chikungunya et Zika, ainsi que les virus posant un risque d'émergence future.
Dans un deuxième chapitre, nous décrirons en détail les modèles utilisés pour décrire et analyser les épidémies de maladies vectorielles, partant de la formulation classique du modèle de Ross-Macdonald, et suivant l'évolution des concepts et techniques au cours des dernières décennies. 
Nous y trouverons l'occasion de présenter les résultats de plusieurs travaux de modélisation de maladies transmises par les moustiques {\em Aedes}.

Au chapitre 3, nous présenterons le premier article de cette thèse.
Ayant mis en évidence les conséquences déjà observées de l'invasion des moustiques du genre {\em Aedes}, ainsi que le risque de nouvelles émergences à moyen terme, l'analyse des caractéristiques communes aux maladies transmises par ces vecteurs pourrait permettre de mieux anticiper et contrôler les émergences futures.
Nous avons donc conduit un travail d'épidémiologie comparative de dix-huit épidémies de chikungunya et de Zika survenues entre 2013 et 2016 en Polynésie française et aux Antilles françaises.
Le principal objectif de cette analyse était de quantifier les relations entre ces épidémies et de distinguer les influences respectives du virus, des conditions météorologiques et des caractéristiques propres aux populations et aux territoires sur les dynamiques épidémiques.
Au chapitre 4, nous présentons un second article, s'attachant au problème de la prédiction d'épidémies émergentes dès le stade précoce.
Ayant quantifié les relations entre les dynamiques épidémiques du chikungunya et du Zika, l'objectif était de vérifier si il était possible de tirer parti de ces connaissances pour améliorer la qualité des prédictions épidémiques.

